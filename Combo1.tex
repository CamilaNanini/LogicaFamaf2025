\documentclass[a4paper,12pt]{article}
\usepackage[spanish]{babel}
\usepackage[utf8]{inputenc}
\usepackage[T1]{fontenc}
\usepackage{amsmath, amssymb, amsthm}
\usepackage{enumitem}

\title{Combos de Definiciones}
\author{Camila Nanini}

\newtheorem{theorem}{Teorema}
\newtheorem{lemma}{Lema}
\newcommand{\deduce}[2]{(#1, #2) \vdash}
\newcommand{\incon}{es inconsistente}

\begin{document}

\maketitle

\section*{Combo 1}

\begin{theorem}[del Filtro Primo]
Sea $(L, s, i)$ un reticulado terna distributivo y $F$ un filtro. Supongamos $x_0 \in L - F$. Entonces hay un filtro primo $P$ tal que
\[
x_0 \notin P \quad \text{y} \quad F \subseteq P.
\]

\end{theorem}

\begin{lemma}[Propiedades básicas de la consistencia]
Sea $(\Sigma, \tau)$ una teoría.
\begin{enumerate}[label=(\arabic*)]
    \item Si $(\Sigma, \tau)$ es inconsistente, entonces $(\Sigma, \tau) \vdash \varphi$, para toda sentencia $\varphi$.
    \item Si $(\Sigma, \tau)$ es consistente y $(\Sigma, \tau) \vdash \varphi$, entonces $(\Sigma \cup \{\varphi\}, \tau)$ es consistente.
    \item Si $(\Sigma, \tau) \not\vdash \neg\varphi$, entonces $(\Sigma \cup \{\varphi\}, \tau)$ es consistente.
\end{enumerate}

\begin{proof}

(1) Si $(\Sigma, \tau)$ es inconsistente, entonces por definición tenemos que 
$(\Sigma, \tau) \vdash \psi \wedge \neg \psi$ para alguna sentencia $\psi$. 
Dada una sentencia cualquiera $\varphi$, tenemos que $\varphi$ se deduce por la regla del absurdo a partir de 
$\psi \wedge \neg \psi$, con lo cual (2) del Lema de propiedades básicas de $\vdash$ nos dice que 
$(\Sigma, \tau) \vdash \varphi$.


(2) Supongamos $(\Sigma, \tau)$ es consistente y $(\Sigma, \tau) \vdash \varphi$. 
Si $(\Sigma \cup \{\varphi\}, \tau)$ fuera inconsistente, entonces 
$(\Sigma \cup \{\varphi\}, \tau) \vdash \psi \wedge \neg \psi$, 
para alguna sentencia $\psi$, lo cual por (1) del Lema de propiedades básicas de $\vdash$ nos diría que 
$(\Sigma, \tau) \vdash \psi \wedge \neg \psi$, 
es decir, nos diría que $(\Sigma, \tau)$ es inconsistente.

(3)  Veamos por \textbf{contradicción}. Supongamos que $(\Sigma \cup \{\varphi\}, \tau)$ \incon.
Esto significa que
\[
    \deduce{\Sigma \cup \{\varphi\}}{\tau} (\psi \land \neg \psi) \quad \text{para algún } \psi.
\]
Por \textbf{(3)} del \textbf{lema de propiedades básicas de $\vdash$} tenemos
\[
    \deduce{\Sigma}{\tau} \varphi \to (\psi \land \neg \psi)
\]
Por \textbf{regla del absurdo} tenemos
\[
    \deduce{\Sigma}{\tau} \neg \varphi
\]
Lo cual \textbf{contradice la hipótesis}.


\end{proof}

\end{lemma}

\section*{Auxiliares de Demostraciones}
\begin{enumerate}
    \item Lema de propiedades básicas de $\vdash$: Sea $(\Sigma, \tau)$ una teoría.

  \begin{enumerate}
    \item (\textbf{Uso de Teoremas}) Si 
    $(\Sigma, \tau) \vdash \varphi_1, \ldots, \varphi_n$ 
    y 
    $(\Sigma \cup \{\varphi_1, \ldots, \varphi_n\}, \tau) \vdash \varphi$, 
    entonces 
    $(\Sigma, \tau) \vdash \varphi$.

    \item Supongamos 
    $(\Sigma, \tau) \vdash \varphi_1, \ldots, \varphi_n$. 
    Si $R$ es una regla distinta de \textsc{Generalización} y \textsc{Elección}, 
    y $\varphi$ se deduce de $\varphi_1, \ldots, \varphi_n$ por la regla $R$, 
    entonces 
    $(\Sigma, \tau) \vdash \varphi$.

    \item $(\Sigma, \tau) \vdash (\varphi \to \psi)$ 
    si y sólo si 
    $(\Sigma \cup \{\varphi\}, \tau) \vdash \psi$.
  \end{enumerate}

\end{enumerate}

\end{document}

