\documentclass[a4paper,12pt]{article}
\usepackage[spanish]{babel}
\usepackage[utf8]{inputenc}
\usepackage[T1]{fontenc}
\usepackage{amsmath, amssymb, amsthm}
\usepackage{enumitem}

\title{Combos de Definiciones}
\author{Camila Nanini}

\newtheorem{theorem}{Teorema}
\newtheorem{lemma}{Lema}
\newcommand{\deduce}[2]{(#1, #2) \vdash}
\newcommand{\incon}{es inconsistente}

\begin{document}

\maketitle

\section*{Combo 1}

\begin{theorem}[del Filtro Primo]
Sea $(L, s, i)$ un reticulado terna distributivo y $F$ un filtro. Supongamos $x_0 \in L - F$. Entonces hay un filtro primo $P$ tal que
\[
x_0 \notin P \quad \text{y} \quad F \subseteq P.
\]

\begin{proof}
Sea 
\[
\mathcal{F} = \{ F_1 : F_1 \text{ es un filtro}, \; x_0 \notin F_1 \text{ y } F \subseteq F_1 \}.
\]

Nótese que $\mathcal{F} \neq \emptyset$, por lo cual $(\mathcal{F}, \subseteq)$ es un poset.  
Veamos que cada cadena en $(\mathcal{F}, \subseteq)$ tiene una cota superior. Sea $C$ una cadena.  
Si $C = \emptyset$, entonces cualquier elemento de $\mathcal{F}$ es cota de $C$.  
Supongamos entonces $C \neq \emptyset$. Sea
\[
G = \{ x : x \in F_1 \text{ para algún } F_1 \in C \}.
\]

Veamos que $G$ es un filtro. Es claro que $G$ es no vacío.  
Supongamos que $x, y \in G$. Sean $F_1, F_2 \in C$ tales que $x \in F_1$ y $y \in F_2$.  
Si $F_1 \subseteq F_2$, entonces, ya que $F_2$ es un filtro, tenemos que 
\[
x \, i \, y \in F_2 \subseteq G.
\]
Si $F_2 \subseteq F_1$, entonces 
\[
x \, i \, y \in F_1 \subseteq G.
\]
Ya que $C$ es una cadena, tenemos que siempre $x \, i \, y \in G$.  
De forma análoga se prueba la propiedad restante, por lo cual $G$ es un filtro.  
Además $x_0 \notin G$, por lo que $G \in \mathcal{F}$ es cota superior de $C$.

Por el **Lema de Zorn**, $(\mathcal{F}, \subseteq)$ tiene un elemento maximal $P$.  
Veamos que $P$ es un **filtro primo**.  
Supongamos $x \, s \, y \in P$ y $x, y \notin P$.  
Nótese que $[P \cup \{x\})$ es un filtro el cual contiene propiamente a $P$.  
Entonces, ya que $P$ es un elemento maximal de $(\mathcal{F}, \subseteq)$, tenemos que 
\[
x_0 \in [P \cup \{x\}).
\]
Análogamente, 
\[
x_0 \in [P \cup \{y\}).
\]
Ya que $x_0 \in [P \cup \{x\})$, tenemos que existen elementos $p_1, \ldots, p_n \in P$ tales que
\[
x_0 \ge p_1 \, i \, \cdots \, i \, p_n \, i \, x
\]
(se deja como ejercicio justificar esto usar la def de Filtro generado por S).  
Ya que $x_0 \in [P \cup \{y\})$, existen elementos $q_1, \ldots, q_m \in P$ tales que
\[
x_0 \ge q_1 \, i \, \cdots \, i \, q_m \, i \, y.
\]
Si llamamos $p$ al siguiente elemento de $P$:
\[
p = p_1 \, i \, \cdots \, i \, p_n \, i \, q_1 \, i \, \cdots \, i \, q_m,
\]
tenemos que
\[
x_0 \ge p \, i \, x, \qquad x_0 \ge p \, i \, y.
\]
Se tiene entonces que
\[
x_0 \ge (p \, i \, x) \, s \, (p \, i \, y) = p \, i \, (x \, s \, y) \in P,
\]
lo cual es absurdo ya que $x_0 \notin P$.
\end{proof}
\end{theorem}

\begin{lemma}[Propiedades básicas de la consistencia]
Sea $(\Sigma, \tau)$ una teoría.
\begin{enumerate}[label=(\arabic*)]
    \item Si $(\Sigma, \tau)$ es inconsistente, entonces $(\Sigma, \tau) \vdash \varphi$, para toda sentencia $\varphi$.
    \item Si $(\Sigma, \tau)$ es consistente y $(\Sigma, \tau) \vdash \varphi$, entonces $(\Sigma \cup \{\varphi\}, \tau)$ es consistente.
    \item Si $(\Sigma, \tau) \not\vdash \neg\varphi$, entonces $(\Sigma \cup \{\varphi\}, \tau)$ es consistente.
\end{enumerate}

\begin{proof}

(1) Si $(\Sigma, \tau)$ es inconsistente, entonces por definición tenemos que 
$(\Sigma, \tau) \vdash \psi \wedge \neg \psi$ para alguna sentencia $\psi$. 
Dada una sentencia cualquiera $\varphi$, tenemos que $\varphi$ se deduce por la regla del absurdo a partir de 
$\psi \wedge \neg \psi$, con lo cual (2) del Lema de propiedades básicas de $\vdash$ nos dice que 
$(\Sigma, \tau) \vdash \varphi$.


(2) Supongamos $(\Sigma, \tau)$ es consistente y $(\Sigma, \tau) \vdash \varphi$. 
Si $(\Sigma \cup \{\varphi\}, \tau)$ fuera inconsistente, entonces 
$(\Sigma \cup \{\varphi\}, \tau) \vdash \psi \wedge \neg \psi$, 
para alguna sentencia $\psi$, lo cual por (1) del Lema de propiedades básicas de $\vdash$ nos diría que 
$(\Sigma, \tau) \vdash \psi \wedge \neg \psi$, 
es decir, nos diría que $(\Sigma, \tau)$ es inconsistente.

(3)  Veamos por \textbf{contradicción}. Supongamos que $(\Sigma \cup \{\varphi\}, \tau)$ \incon.
Esto significa que
\[
    \deduce{\Sigma \cup \{\varphi\}}{\tau} (\psi \land \neg \psi) \quad \text{para algún } \psi.
\]
Por \textbf{(3)} del \textbf{lema de propiedades básicas de $\vdash$} tenemos
\[
    \deduce{\Sigma}{\tau} \varphi \to (\psi \land \neg \psi)
\]
Por \textbf{regla del absurdo} tenemos
\[
    \deduce{\Sigma}{\tau} \neg \varphi
\]
Lo cual \textbf{contradice la hipótesis}.


\end{proof}

\end{lemma}

\section*{Auxiliares de Demostraciones}
\begin{enumerate}
    \item Lema de propiedades básicas de $\vdash$: Sea $(\Sigma, \tau)$ una teoría.

  \begin{enumerate}
    \item (\textbf{Uso de Teoremas}) Si 
    $(\Sigma, \tau) \vdash \varphi_1, \ldots, \varphi_n$ 
    y 
    $(\Sigma \cup \{\varphi_1, \ldots, \varphi_n\}, \tau) \vdash \varphi$, 
    entonces 
    $(\Sigma, \tau) \vdash \varphi$.

    \item Supongamos 
    $(\Sigma, \tau) \vdash \varphi_1, \ldots, \varphi_n$. 
    Si $R$ es una regla distinta de \textsc{Generalización} y \textsc{Elección}, 
    y $\varphi$ se deduce de $\varphi_1, \ldots, \varphi_n$ por la regla $R$, 
    entonces 
    $(\Sigma, \tau) \vdash \varphi$.

    \item $(\Sigma, \tau) \vdash (\varphi \to \psi)$ 
    si y sólo si 
    $(\Sigma \cup \{\varphi\}, \tau) \vdash \psi$.
    \end{enumerate}

    \item {(Lema de Zorn)} Sea $(P, \leq)$ un poset y supongamos que cada cadena de $(P, \leq)$ tiene al menos una cota superior. Entonces, hay un elemento maximal en $(P, \leq)$.
    \item Filtro generado por $S$: Dado un conjunto $S \subseteq L$, denotemos con $[S)$ el siguiente conjunto:
    \[
    [S) = \{\, y \in L : y \geq s_1 \, i \, \cdots \, i \, s_n \text{, para algunos } s_1, \ldots, s_n \in S,\; n \geq 1 \,\}.
    \]



\end{enumerate}

\end{document}

