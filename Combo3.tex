\documentclass[a4paper,12pt]{article}
\usepackage[spanish]{babel}
\usepackage[utf8]{inputenc}
\usepackage[T1]{fontenc}
\usepackage{amsmath, amssymb, amsthm}
\usepackage{enumitem}

\title{Combos de Definiciones}
\author{Camila Nanini}

\newtheorem{theorem}{Teorema}
\newtheorem{lemma}{Lema}
\newcommand{\deduce}[2]{(#1, #2) \vdash}
\newcommand{\incon}{es inconsistente}

\begin{document}

\maketitle

\section*{Combo 3}

\begin{theorem}[Lectura única de términos]
Dado $t \in T^\tau$, se da una de las siguientes:
\begin{enumerate}[label=(\arabic*)]
    \item $t \in Var \cup C$
    \item Hay únicos $n \ge 1$, $f \in F_n$, $t_1, \ldots, t_n \in T^\tau$ tales que $t = f(t_1, \ldots, t_n)$.
\end{enumerate}
\begin{proof}
\noindent
Por la definición de $T^\tau$ está claro que vale sin la unicidad (1)
En virtud del Lema de Menú de términos solo nos falta probar la unicidad en el punto (2). 
Supongamos
\[
t = f(t_1,\ldots,t_n) = g(s_1,\ldots,s_m)
\]
con $n,m\ge 1$, $f\in F_n$, $g\in F_m$, $t_1,\ldots,t_n,s_1,\ldots,s_m\in T_\tau$.
Nótese que $f=g$. Es decir, $n=m=a(f)$. 

\noindent
Nótese que $t_1$ es tramo inicial de $s_1$ o $s_1$ es tramo inicial de $t_1$, lo cual, por el lema de mordisqueo de términos, nos dice que $t_1=s_1$. Con el mismo razonamiento se prueba que necesariamente
\[
t_2=s_2,\ \ldots,\ t_n=s_n.
\]
\end{proof}
\end{theorem}

\begin{lemma}
Supongamos que $F: A \to B$ es un isomorfismo. Sea $\varphi \in F^\tau$. Entonces
\[
A \vDash \varphi[(a_1, a_2, \ldots)] \text{ si y sólo si } B \vDash \varphi[(F(a_1), F(a_2), \ldots)]
\]
para cada $(a_1, a_2, \ldots) \in A^N$. En particular, $A$ y $B$ satisfacen las mismas sentencias de tipo $\tau$.
\begin{proof}
Para $\vec{a} = (a_1, a_2, \ldots) \in A^N$, denotemos $(F(a_1), F(a_2), \ldots)$ con $F(\vec{a})$.  
Procedemos por inducción.

\medskip
\noindent
\textbf{Teo$_k$:} Supongamos que $F: A \to B$ es un isomorfismo. Sea $\varphi \in F^\tau_k$. Entonces
\[
A \vDash \varphi[\vec{a}] \text{ sii } B \vDash \varphi[F(\vec{a})],
\]
para cada $(a_1,a_2,\ldots) \in A^N$.

\medskip
\noindent
\textbf{Prueba de Teo$_0$.}  
Hay dos casos.  
\smallskip

\noindent
\textbf{Caso} $\varphi = r(t_1, \ldots, t_n)$, con $n \ge 1$, $r \in R_n$ y $t_1, \ldots, t_n \in T^\tau$.  
Tenemos entonces:
\[
\begin{aligned}
A \vDash \varphi[\vec{a}]
&\text{ sii } (t_1^A[\vec{a}], \ldots, t_n^A[\vec{a}]) \in r^A && \text{(def. de $\vDash$)}\\
&\text{ sii } (F(t_1^A[\vec{a}]), \ldots, F(t_n^A[\vec{a}])) \in r^B && \text{($F$ es iso)}\\
&\text{ sii } (t_1^B[F(\vec{a})], \ldots, t_n^B[F(\vec{a})]) \in r^B && \text{(por lema auxiliar 2)}\\
&\text{ sii } B \vDash \varphi[F(\vec{a})].
\end{aligned}
\]

\textbf{Caso} $\varphi = (t \equiv s)$ con $t,s \in T^\tau$. Tenemos ahora que 
\[
\begin{aligned}
A \vDash \varphi[\vec{a}]
&\text{ sii } (t^A[\vec{a}] \equiv s^A[\vec{a}]) && \text{(def. de $\vDash$)}\\
&\text{ sii } (F(t^A[\vec{a}]) \equiv F(s^A[\vec{a}])) && \text{($F$ es iso)}\\
&\text{ sii } (t^B[F(\vec{a})] \equiv s^B[F(\vec{a})]) && \text{(por lema auxiliar 2)}\\
&\text{ sii } B \vDash \varphi[F(\vec{a})].
\end{aligned}
\]
\medskip
\noindent
Veamos ahora que Teo$_k$ implica Teo$_{k+1}$.  
Supongamos que vale Teo$_k$. Probaremos que entonces vale Teo$_{k+1}$.

\smallskip
Si $\varphi \in F^\tau_k$, podemos aplicar directamente Teo$_k$.  
Supongamos entonces que $\varphi \in F^\tau_{k+1} - F^\tau_k$.  
Por el Lema de Lectura Única de fórmulas, hay varios casos.

\medskip
\noindent
\textbf{Caso} $\varphi = (\varphi_1 \lor \varphi_2)$, con $\varphi_1, \varphi_2 \in F^\tau_k$. Entonces:
\[
\begin{aligned}
A \vDash \varphi[\vec{a}]
&\text{ sii } A \vDash \varphi_1[\vec{a}] \text{ o } A \vDash \varphi_2[\vec{a}] && \text{(def. de $\vDash$)}\\
&\text{ sii } B \vDash \varphi_1[F(\vec{a})] \text{ o } B \vDash \varphi_2[F(\vec{a})] && \text{(Teo$_k$)}\\
&\text{ sii } B \vDash \varphi[F(\vec{a})] && \text{(def. de $\vDash$)}.
\end{aligned}
\]

Los casos $\varphi = (\varphi_1 \land \varphi_2)$, $\varphi = (\varphi_1 \to \varphi_2)$, 
$\varphi = (\varphi_1 \leftrightarrow \varphi_2)$ y $\varphi = \neg \varphi_1$ son análogos al anterior.

\medskip
\noindent
\textbf{Caso} $\varphi = \forall x_j \varphi_1$, con $\varphi_1 \in F^\tau_k$.  
Veamos cada implicación por separado.

\smallskip
\noindent
Supongamos $A \vDash \varphi[\vec{a}]$.  
Entonces, por la definición de $\vDash$, se tiene que 
\[
A \vDash \varphi_1[\downarrow^a_j(\vec{a})], \quad \text{para todo } a \in A.
\]
Por Teo$_k$ tenemos que 
\[
B \vDash \varphi_1[F(\downarrow^a_j(\vec{a}))], \quad \text{para todo } a \in A.
\]
Pero como
\[
F(\downarrow^a_j(\vec{a})) = \downarrow^{F(a)}_j(F(\vec{a})),
\]
tenemos que 
\[
B \vDash \varphi_1[\downarrow^{F(a)}_j(F(\vec{a}))], \quad \text{para todo } a \in A.
\]
Como $F$ es sobreyectiva, obtenemos que 
\[
B \vDash \varphi_1[\downarrow^b_j(F(\vec{a}))], \quad \text{para todo } b \in B.
\]
Ahora, por la definición de $\vDash$, tenemos que 
\[
B \vDash \forall x_j \varphi_1[F(\vec{a})],
\]
es decir, $B \vDash \varphi[F(\vec{a})]$.

\smallskip
\noindent
Recíprocamente, supongamos que $B \vDash \varphi[F(\vec{a})]$.  
La definición de $\vDash$ nos dice que 
\[
B \vDash \varphi_1[\downarrow^b_j(F(\vec{a}))], \quad \text{para todo } b \in B.
\]
Obviamente, esto implica que 
\[
B \vDash \varphi_1[\downarrow^{F(a)}_j(F(\vec{a}))], \quad \text{para todo } a \in A.
\]
Pero como 
\[
\downarrow^{F(a)}_j(F(\vec{a})) = F(\downarrow^a_j(\vec{a})),
\]
tenemos que 
\[
B \vDash \varphi_1[F(\downarrow^a_j(\vec{a}))], \quad \text{para todo } a \in A.
\]
Por Teo$_k$, se sigue que 
\[
A \vDash \varphi_1[\downarrow^a_j(\vec{a})], \quad \text{para todo } a \in A,
\]
lo cual, por la definición de $\vDash$, nos dice que $A \vDash \varphi[\vec{a}]$.

\medskip
\noindent
El caso $\varphi = \exists x_j \varphi_1$ es análogo al anterior.
\end{proof}
\end{lemma}

\begin{theorem}
Sea $T = (\Sigma, \tau)$ una teoría. Entonces
\[
(S^\tau / \dashv\vdash_T, s^T, i^T, c^T, 0^T, 1^T)
\]
es un álgebra de Boole. \\

\textbf{Pruebe sólo el ítem (6).}
\begin{proof}
Veamos que
\[
[\varphi_1]_T \, s^T \, \big( [\varphi_2]_T \, s^T \, [\varphi_3]_T \big)
= \big( [\varphi_1]_T \, s^T \, [\varphi_2]_T \big) \, s^T \, [\varphi_3]_T,
\]
cualesquiera sean $\varphi_1, \varphi_2, \varphi_3 \in S^\tau$.

Sean $\varphi_1, \varphi_2, \varphi_3 \in S^\tau$ fijas.  
Por la definición de la operación $s^T$ tenemos que:
\[
\begin{aligned}
[\varphi_1]_T \, s^T \, \big( [\varphi_2]_T \, s^T \, [\varphi_3]_T \big)
&= [\varphi_1]_T \, s^T \, [(\varphi_2 \vee \varphi_3)]_T \\
&= [(\varphi_1 \vee (\varphi_2 \vee \varphi_3))]_T, \\[6pt]
\big( [\varphi_1]_T \, s^T \, [\varphi_2]_T \big) \, s^T \, [\varphi_3]_T
&= [(\varphi_1 \vee \varphi_2)]_T \, s^T \, [\varphi_3]_T \\
&= [((\varphi_1 \vee \varphi_2) \vee \varphi_3)]_T.
\end{aligned}
\]

\noindent
Por tanto, debemos probar que
\[
[(\varphi_1 \vee (\varphi_2 \vee \varphi_3))]_T
= [((\varphi_1 \vee \varphi_2) \vee \varphi_3)]_T,
\]
es decir, que
\[
T \vdash 
\big( (\varphi_1 \vee (\varphi_2 \vee \varphi_3)) 
\leftrightarrow 
((\varphi_1 \vee \varphi_2) \vee \varphi_3) \big).
\]

\noindent
Nótese que, por (2) del lema de propiedades básicas de $\vdash$, basta con probar que:
\[
T \vdash 
\big( (\varphi_1 \vee (\varphi_2 \vee \varphi_3))
\rightarrow 
((\varphi_1 \vee \varphi_2) \vee \varphi_3) \big),
\qquad
T \vdash 
\big( ((\varphi_1 \vee \varphi_2) \vee \varphi_3)
\rightarrow 
(\varphi_1 \vee (\varphi_2 \vee \varphi_3)) \big).
\]

\noindent
A continuación damos una prueba formal de
\[
(\varphi_1 \vee (\varphi_2 \vee \varphi_3))
\rightarrow 
((\varphi_1 \vee \varphi_2) \vee \varphi_3)
\text{ en } T,
\]

\begin{enumerate}[label=\arabic*.]
    \item $(\varphi_1 \vee (\varphi_2 \vee \varphi_3))$ \hfill \textit{Hipótesis 1}
    \item $\varphi_1$ \hfill \textit{Hipótesis 2}
    \item $(\varphi_1 \vee \varphi_2)$ \hfill \textit{Introducción de $\vee$ (2)}
    \item $((\varphi_1 \vee \varphi_2) \vee \varphi_3)$ \hfill \textit{Tesis 2, Introducción de $\vee$ (3)}
    \item $\varphi_1 \rightarrow ((\varphi_1 \vee \varphi_2) \vee \varphi_3)$ \hfill \textit{Conclusión}
    \item $(\varphi_2 \vee \varphi_3)$ \hfill \textit{Hipótesis 3}
    \item $\varphi_2$ \hfill \textit{Hipótesis 4}
    \item $(\varphi_1 \vee \varphi_2)$ \hfill \textit{Introducción de $\vee$ (6)}
    \item $((\varphi_1 \vee \varphi_2) \vee \varphi_3)$ \hfill \textit{Tesis 4, Introducción de $\vee$ (7)}
    \item $\varphi_2 \rightarrow ((\varphi_1 \vee \varphi_2) \vee \varphi_3)$ \hfill \textit{Conclusión}
    \item $\varphi_3$ \hfill \textit{Hipótesis 5}
    \item $((\varphi_1 \vee \varphi_2) \vee \varphi_3)$ \hfill \textit{Tesis 5, Introducción de $\vee$ (11)}
    \item $\varphi_3 \rightarrow ((\varphi_1 \vee \varphi_2) \vee \varphi_3)$ \hfill \textit{Conclusión}
    \item $((\varphi_1 \vee \varphi_2) \vee \varphi_3)$ \hfill \textit{Tesis 3, División por casos (6, 10, 13)}
    \item $(\varphi_2 \vee \varphi_3) \rightarrow ((\varphi_1 \vee \varphi_2) \vee \varphi_3)$ \hfill \textit{Conclusión}
    \item $((\varphi_1 \vee \varphi_2) \vee \varphi_3)$ \hfill \textit{Tesis 1, División por casos (1, 5, 15)}
    \item $(\varphi_1 \vee (\varphi_2 \vee \varphi_3)) \rightarrow ((\varphi_1 \vee \varphi_2) \vee \varphi_3)$ \hfill \textit{Conclusión}
\end{enumerate}

\noindent
A continuación damos una prueba formal de
\[
((\varphi_1 \vee \varphi_2) \vee \varphi_3)
\rightarrow 
(\varphi_1 \vee (\varphi_2 \vee \varphi_3))
\text{ en } T,
\]

\begin{enumerate}[label=\arabic*.]
    \item $((\varphi_1 \vee \varphi_2) \vee \varphi_3)$ \hfill \textit{Hipótesis 1}
    \item $(\varphi_1 \vee \varphi_2)$ \hfill \textit{Hipótesis 2}
    \item $\varphi_1$ \hfill \textit{Hipótesis 3}
    \item $(\varphi_1 \vee (\varphi_2 \vee \varphi_3))$ \hfill \textit{Tesis 3 Disjunción Introducción 3}
    \item $\varphi_1 \rightarrow (\varphi_1 \vee (\varphi_2 \vee \varphi_3))$ \hfill \textit{Conclusión}
    \item $\varphi_2$ \hfill \textit{Hipótesis 4}
    \item $(\varphi_2 \vee \varphi_3)$ \hfill \textit{Disjunción Introducción 6}
    \item $(\varphi_1 \vee (\varphi_2 \vee \varphi_3))$ \hfill \textit{Tesis 4 Disjunción Introducción 7}
    \item $\varphi_2 \rightarrow (\varphi_1 \vee (\varphi_2 \vee \varphi_3))$ \hfill \textit{Conclusión}
    \item $(\varphi_1 \vee (\varphi_2 \vee \varphi_3))$ \hfill \textit{Tesis 2 División por casos 2,5,9}
    \item $(\varphi_1 \vee \varphi_2) \rightarrow (\varphi_1 \vee (\varphi_2 \vee \varphi_3))$ \hfill \textit{Conclusión}
    \item $\varphi_3$ \hfill \textit{Hipótesis 5}
    \item $(\varphi_2 \vee \varphi_3)$ \hfill \textit{Disjunción 12}
    \item $(\varphi_1 \vee (\varphi_2 \vee \varphi_3))$ \hfill \textit{Tesis 5 Disjunción 13} 
    \item $\varphi_3 \rightarrow (\varphi_1 \vee (\varphi_2 \vee \varphi_3))$ \hfill \textit{Conclusión}
    \item $(\varphi_1 \vee (\varphi_2 \vee \varphi_3))$ \hfill \textit{Tesis 1 División por casos 1,11,15}
    \item $((\varphi_1 \vee \varphi_2) \vee \varphi_3) \rightarrow (\varphi_1 \vee (\varphi_2 \vee \varphi_3))$ \hfill \textit{Conclusión} 
\end{enumerate}

\end{proof}
\end{theorem}

\section*{Auxiliares de Demostraciones}
\begin{enumerate}
    \item Lema de menú de términos: Supongamos $t \in T_k^\tau$, con $k \ge 1$. Entonces se da alguna de las siguientes:
    \begin{enumerate}[label=(\alph*)]
        \item $t \in Var \cup C$.
        \item $t = f(t_1, \ldots, t_n)$, con $f \in F_n$, $n \ge 1$ y $t_1, \ldots, t_n \in T_{k-1}^\tau$.
    \end{enumerate}

    \item Dado un tipo $\tau$, definamos recursivamente los conjuntos de palabras $T^\tau_k$, con $k \ge 0$, de la siguiente manera:
    \[
    T^\tau_0 = Var \,\cup\, C
    \]
    \[
    T^\tau_{k+1}
    = T^\tau_k \,\cup\, 
    \{\, f(t_1, \ldots, t_n) : f \in F_n,\ n \ge 1,\ \text{y } t_1, \ldots, t_n \in T^\tau_k \,\}.
    \]
    Sea
    \[
    T_\tau = \bigcup_{k \ge 0} T^\tau_k.
    \]
    Los elementos de $T^\tau$ serán llamados \emph{términos de tipo} $\tau$.

    \item Lema auxiliar 2: Sea $F : A \to B$ un homomorfismo. Entonces
    \[
    F\big(t^A[(a_1, a_2, \ldots)]\big) = t^B\big[(F(a_1), F(a_2), \ldots)\big]
    \]
    para cada $t \in T^\tau$, $(a_1, a_2, \ldots) \in A^N$.
    \item Lema de Mordisqueo de terminos: Sean $s, t \in T^\tau$ y supongamos que hay palabras $x, y, z$, con $y \ne \varepsilon$, tales que 
    $s = xy$ y $t = yz$. Entonces $x = z = \varepsilon$ o bien $s, t \in C$. 

    En particular, si un término es tramo inicial o final de otro término, entonces dichos términos son iguales.

    \item Lema de Lectura Única de fórmulas: Dada $\varphi \in F^\tau$ se da una y sólo una de las siguientes:

    \begin{enumerate}
        \item $\varphi = (t \equiv s)$, con $t, s \in T^\tau$.
        \item $\varphi = r(t_1, \ldots, t_n)$, con $r \in R_n$, $t_1, \ldots, t_n \in T^\tau$.
        \item $\varphi = (\varphi_1 \, \eta \, \varphi_2)$, con $\eta \in \{\wedge, \vee, \to, \leftrightarrow\}$, $\varphi_1, \varphi_2 \in F^\tau$.
        \item $\varphi = \neg \varphi_1$, con $\varphi_1 \in F^\tau$.
        \item $\varphi = Qv\varphi_1$, con $Q \in \{\forall, \exists\}$, $\varphi_1 \in F^\tau$ y $v \in Var$.
    \end{enumerate}

    Más aún, en los puntos (1), (2), (3), (4) y (5) tales descomposiciones son únicas.

    \item Lema de propiedades básicas de $\vdash$:Sea $(\Sigma, \tau)$ una teoría.

    \begin{enumerate}
    \item (\textbf{Uso de Teoremas}) Si 
    $(\Sigma, \tau) \vdash \varphi_1, \ldots, \varphi_n$ 
    y 
    $(\Sigma \cup \{\varphi_1, \ldots, \varphi_n\}, \tau) \vdash \varphi$, 
    entonces 
    $(\Sigma, \tau) \vdash \varphi$.

    \item Supongamos 
    $(\Sigma, \tau) \vdash \varphi_1, \ldots, \varphi_n$. 
    Si $R$ es una regla distinta de \textsc{Generalización} y \textsc{Elección}, 
    y $\varphi$ se deduce de $\varphi_1, \ldots, \varphi_n$ por la regla $R$, 
    entonces 
    $(\Sigma, \tau) \vdash \varphi$.

    \item $(\Sigma, \tau) \vdash (\varphi \to \psi)$ 
    si y sólo si 
    $(\Sigma \cup \{\varphi\}, \tau) \vdash \psi$.
  \end{enumerate}
\end{enumerate}

\end{document}

