\documentclass[a4paper,12pt]{article}
\usepackage[spanish]{babel}
\usepackage[utf8]{inputenc}
\usepackage[T1]{fontenc}
\usepackage{amsmath, amssymb, amsthm}
\usepackage{enumitem}

\title{Combo 6}
\author{Camila Nanini}

\newtheorem{theorem}{Teorema}
\newtheorem{lemma}{Lema}
\newcommand{\deduce}[2]{(#1, #2) \vdash}
\newcommand{\incon}{es inconsistente}

\begin{document}

\maketitle

\section*{Combo 6}

\begin{theorem}[Teorema de Completitud]
Sea $T = (\Sigma, \tau)$ una teoría de primer orden.  
Si $T \models \varphi$, entonces $T \vdash \varphi$.
\end{theorem}

Haga solo el caso en que $\tau$ tiene una cantidad infinita de nombres de constantes que no ocurren en las sentencias de $\Sigma$.  
En la exposición de la prueba no es necesario que demuestre los ítems: (1), (2), (3) y (4).

\begin{proof}
Primero probaremos completitud para el caso en que $\tau$ tiene una cantidad infinita
de nombres de constante que no ocurren en las sentencias de $\Sigma$.
Lo probaremos por el absurdo: supongamos que hay una sentencia $\varphi_0$ tal que
$T \models \varphi_0$ y $T \nvdash \varphi_0$.

Nótese que, ya que $T \nvdash \varphi_0$, tenemos que
$[\varphi_0]_T \ne 1^T = \{\varphi \in S^\tau : T \vdash \varphi\}$
O sea que $[\lnot \varphi_0]_T \ne 0^T$.

Por el Lema de enumeracion hay una infinitupla $(\gamma_1,\gamma_2,\dots)\in F^{\tau \mathbb N}$ tal que:

\begin{itemize}
    \item $|Li(\gamma_j)| \le 1$ para cada $j = 1,2,\dots$
    \item Si $|Li(\gamma)| \le 1$, entonces $\gamma = \gamma_j$ para algún $j \in \mathbb N$.
\end{itemize}

Para cada $j\in\mathbb N$, sea $w_j \in Var$ tal que $Li(\gamma_j) \subseteq \{w_j\}$.
Para cada $j$, declaremos $\gamma_j \; {:=}_d\; \gamma_j(w_j)$.

Nótese que por el Lema del Infimo tenemos que
\[
\inf\bigl(\{[\gamma_j(t)]_T : t \in T^\tau_c\}\bigr)
= [\forall w_j\, \gamma_j(w_j)]_T,
\qquad j=1,2,\dots
\]

Por el Teorema de Rasiowa–Sikorski, existe un filtro primo $U$ de $A_T$ tal que:

\begin{enumerate}[label=(\alph*)]
    \item $[\lnot \varphi_0]_T \in U$,
    \item Para cada $j\in\mathbb N$,
    \[
    \{[\gamma_j(t)]_T : t\in T^\tau_c\} \subseteq U
    \quad\Longrightarrow\quad
    [\forall w_j\, \gamma_j(w_j)]_T \in U.
    \]
\end{enumerate}

Dado que la infinitupla $(\gamma_1,\gamma_2,\dots)$ cubre todas las fórmulas
con a lo sumo una variable libre, podemos reescribir la propiedad (b) como:

\begin{enumerate}[label=(b$'$)]
    \item Para cada $\varphi {:=}_d \varphi(v) \in F^\tau$, si
    \[
    \{[\varphi(t)]_T : t \in T^\tau_c\} \subseteq U,
    \]
    entonces
    \[
    [\forall v\, \varphi(v)]_T \in U.
    \]
\end{enumerate}

Definimos sobre $T^\tau_c$ la siguiente relación:
\[
t \,\bowtie\, s \quad\text{si y sólo si}\quad [(t \equiv s)]_T \in U.
\]

Entonces se verifica que:

\begin{enumerate}
    \item $\bowtie$ es una relación de equivalencia.

    \item Para cada $\varphi {:=}_d \varphi(v_1,\dots,v_n)\in F_\tau$ y
    $t_1,\dots,t_n,s_1,\dots,s_n \in T^\tau_c$, si
    \[
    t_1 \bowtie s_1,\; t_2 \bowtie s_2,\;\dots,\; t_n \bowtie s_n,
    \]
    entonces
    \[
    [\varphi(t_1,\dots,t_n)]_T \in U
    \quad\Longleftrightarrow\quad
    [\varphi(s_1,\dots,s_n)]_T \in U.
    \]

    \item Para cada $f \in F_n$ y
    $t_1,\dots,t_n,s_1,\dots,s_n \in T^\tau_c$,
    \[
    t_1 \bowtie s_1,\ t_2 \bowtie s_2,\ \dots,\ t_n \bowtie s_n
    \quad\Longrightarrow\quad
    f(t_1,\dots,t_n)\; \bowtie\; f(s_1,\dots,s_n).
    \]

    \item Para cada $t {:=}_d t(v_1,\dots,v_n) \in T^\tau$ y 
    $t_1,\dots,t_n \in T^\tau_c$, tenemos que
    \[
    t^{A_U}[t_1/\!\bowtie,\dots,t_n/\!\bowtie]
    =
    t(t_1,\dots,t_n)/\!\bowtie.
    \]

    \item Para cada $\varphi {:=}_d \varphi(v_1,\dots,v_n)\in F_\tau$ y 
    $t_1,\dots,t_n \in T^\tau_c$, tenemos que
    \[
    A_U \models \varphi[t_1/\!\bowtie,\dots,t_n/\!\bowtie]
    \quad\text{si y sólo si}\quad
    [\varphi(t_1,\dots,t_n)]_T \in U.
    \]
\end{enumerate}
Probaremos (5) por inducción en el $k$ tal que $\varphi \in F^\tau_{k}$. 
El caso $k=0$ es dejado al lector. 

Supongamos que (5) vale para $\varphi \in F^\tau_{k}$.  
Sea ahora $\varphi {:=}_d \varphi(v_1,\dots,v_n) \in F^\tau_{k+1} \setminus F^\tau_k$. 
Hay varios casos:

\subsubsection*{Caso $\varphi = (\varphi_1 \lor \varphi_2)$}

Por la Convención Notacional 6, tenemos 
$\varphi_i {:=}_d \varphi_i(v_1,\dots,v_n)$. Entonces:

\[
\begin{aligned}
A_U \models \varphi[t_1/\!\bowtie,\dots,t_n/\!\bowtie]
&\iff 
A_U \models \varphi_1[t_1/\!\bowtie,\dots,t_n/\!\bowtie]
\text{ o }
A_U \models \varphi_2[t_1/\!\bowtie,\dots,t_n/\!\bowtie] \\
&\iff
[\varphi_1(t_1,\dots,t_n)]_T \in U 
\text{ o }
[\varphi_2(t_1,\dots,t_n)]_T \in U \\
&\iff
[\varphi_1(t_1,\dots,t_n)]_T \, s^T \,
[\varphi_2(t_1,\dots,t_n)]_T \in U \\
&\iff
[(\varphi_1(t_1,\dots,t_n)\lor \varphi_2(t_1,\dots,t_n))]_T \in U \\
&\iff
[\varphi(t_1,\dots,t_n)]_T \in U.
\end{aligned}
\]

\subsubsection*{Caso $\varphi = \forall v\, \varphi_1$ con $v \in Var \setminus \{v_1,\dots,v_n\}$}

Por la Convención Notacional 6, 
$\varphi_1 {:=}_d \varphi_1(v_1,\dots,v_n,v)$. Entonces:

\[
\begin{aligned}
A_U \models \varphi[t_1/\!\bowtie,\dots,t_n/\!\bowtie]
&\iff 
A_U \models \varphi_1[t_1/\!\bowtie,\dots,t_n/\!\bowtie,t/\!\bowtie],
\quad \text{para todo } t\in T_c^\tau \\
&\iff
[\varphi_1(t_1,\dots,t_n,t)]_T \in U,
\quad \text{para todo } t\in T_c^\tau \\
&\iff
[\forall v\,\varphi_1(t_1,\dots,t_n,v)]_T \in U \\
&\iff
[\varphi(t_1,\dots,t_n)]_T \in U.
\end{aligned}
\]

\subsubsection*{Caso $\varphi = \exists v\, \varphi_1$ con $v \in Var \setminus \{v_1,\dots,v_n\}$}

Por la Convención Notacional 6, 
$\varphi_1 {:=}_d \varphi_1(v_1,\dots,v_n,v)$. Entonces:

\[
\begin{aligned}
A_U \models \varphi[t_1/\!\bowtie,\dots,t_n/\!\bowtie]
&\iff 
A_U \models \varphi_1[t_1/\!\bowtie,\dots,t_n/\!\bowtie,t/\!\bowtie],
\quad \text{para algún } t\in T_c^\tau \\
&\iff
[\varphi_1(t_1,\dots,t_n,t)]_T \in U,
\quad \text{para algún } t\in T_c^\tau \\
&\iff
([\varphi_1(t_1,\dots,t_n,t)]_T)^c_T \notin U,
\quad \text{para algún } t\in T_c^\tau \\
&\iff
[\neg\varphi_1(t_1,\dots,t_n,t)]_T \notin U,
\quad \text{para algún } t\in T_c^\tau \\
&\iff
[\forall v\, \neg\varphi_1(t_1,\dots,t_n,v)]_T \notin U \\
&\iff
([\forall v\, \neg\varphi_1(t_1,\dots,t_n,v)]_T)^{c^T} \in U \\
&\iff
[\neg\forall v\, \neg\varphi_1(t_1,\dots,t_n,v)]_T \in U \\
&\iff
[\varphi(t_1,\dots,t_n)]_T \in U.
\end{aligned}
\]

\subsubsection*{Conclusión}

El ítem (5) implica que para cada sentencia $\psi\in S^\tau$:

\[
A_U \models \psi 
\quad\text{si y sólo si}\quad
[\psi]_T \in U.
\]

Por lo tanto:

\[
A_U \models \Sigma
\qquad\text{y}\qquad
A_U \models \neg\varphi_0,
\]

lo cual contradice la suposición de que $T \models \varphi_0$.

\bigskip

Ahora supongamos que $\tau$ es cualquier tipo. 
Sean $s_1$ y $s_2$ símbolos no pertenecientes a la lista
\[
\forall\ \exists\ \neg\ \lor\ \land\ \to\ \leftrightarrow\ ( \ )\ ,\ \equiv\ 
X\ \mathit{0 \ 1 \ \dots\ 9}\ \mathbf{0 \ 1 \ \dots\ 9}
\]
y tales que ninguno ocurra en alguna palabra de $C \cup F \cup R$.

Si $T \models \varphi$, entonces —usando el Lema de Coincidencia— se tiene:

\[
(\Sigma,(C \cup \{s_1s_2s_1,\ s_1s_2s_2s_1,\dots\},F,R,a)) \models \varphi,
\]

por lo cual

\[
(\Sigma,(C \cup \{s_1s_2s_1,\ s_1s_2s_2s_1,\dots\},F,R,a)) \vdash \varphi.
\]

Pero por el Lema de Tipos parecidos, concluimos:

\[
T \vdash \varphi.
\]

\end{proof}

\section*{Auxiliares de Demostraciones}

\begin{enumerate}
    \item \textbf{(Lema de enumeración).} Sea $\tau$ un tipo.  
    Hay una infinitupla $(\gamma_1,\gamma_2,\ldots)\in F^{\tau^{\mathbb{N}}}$ tal que:

    \begin{enumerate}
        \item $\lvert L_i(\gamma_j)\rvert \le 1$, para cada $j = 1,2,\ldots$
        \item Si $\lvert L_i(\gamma)\rvert \le 1$, entonces $\gamma = \gamma_j$ para algún $j\in\mathbb{N}$.
    \end{enumerate}

    \item \textbf{(Lema del Ínfimo).} Sea $T = (\Sigma, \tau)$ una teoría y supongamos que $\tau$ tiene una cantidad infinita 
    de nombres de constante que no ocurren en las sentencias de $\Sigma$. 
    Entonces, para cada fórmula $\varphi =_{d} \varphi(v)$, se tiene que en el álgebra de Lindenbaum $A_T$:

    \[
    [\forall v\, \varphi(v)]_T 
    = 
    \inf\bigl(\{\, [\varphi(t)]_T : t \in T_c^\tau \,\}\bigr).
    \]

    \item \textbf{(Teorema de Rasiowa y Sikorski).} 
    Sea $(B, s, i, c, 0, 1)$ un álgebra de Boole. 
    Sea $a \in B$, $a \neq 0$. 
    Supongamos que $(A_1, A_2, \ldots)$ es una infinitupla de subconjuntos de $B$ tal que existe 
    $\inf(A_j)$ para cada $j = 1,2,\ldots$. 
    Entonces existe un filtro primo $P$ que cumple:

    \begin{enumerate}
        \item[(a)] $a \in P$,
        \item[(b)] Si $P \supseteq A_j$, entonces $\,\inf(A_j) \in P$, para cada $j = 1,2,\ldots$.
    \end{enumerate}
 
    \item \textbf{(Tipos parecidos).} 
    Sean $\tau = (C, F, R, a)$ y $\tau' = (C', F', R', a')$ tipos.

    \begin{enumerate}
        \item Si $C \subseteq C'$, $F \subseteq F'$, $R \subseteq R'$ y $a'|_{F \cup R} = a$, entonces
        \[
        (\Sigma, \tau) \vdash \varphi \quad \Rightarrow \quad (\Sigma, \tau') \vdash \varphi .
        \]

        \item Si $C \subseteq C'$, $F = F'$, $R = R'$ y $a' = a$, entonces
        \[
        (\Sigma, \tau') \vdash \varphi \quad \Rightarrow \quad (\Sigma, \tau) \vdash \varphi,
        \]
        cada vez que $\Sigma \cup \{\varphi\} \subseteq S_\tau$.
    \end{enumerate}

\end{enumerate}

\end{document}
