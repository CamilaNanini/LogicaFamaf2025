\documentclass[a4paper,12pt]{article}
\usepackage[spanish]{babel}
\usepackage[utf8]{inputenc}
\usepackage[T1]{fontenc}
\usepackage{amsmath, amssymb, amsthm}
\usepackage{enumitem}

\title{Combo 5}
\author{Camila Nanini}

\newtheorem{theorem}{Teorema}
\newtheorem{lemma}{Lema}
\newcommand{\deduce}[2]{(#1, #2) \vdash}
\newcommand{\incon}{es inconsistente}

\begin{document}

\maketitle

\section*{Combo 5}

\begin{theorem}[Teorema de Completitud]
Sea $T = (\Sigma, \tau)$ una teoría de primer orden.  
Si $T \models \varphi$, entonces $T \vdash \varphi$.
\end{theorem}

Haga solo el caso en que $\tau$ tiene una cantidad infinita de nombres de constantes que no ocurren en las sentencias de $\Sigma$.  
En la exposición de la prueba no es necesario que demuestre los ítems: (1) y (5).

\begin{proof}
Primero probaremos completitud para el caso en que $\tau$ tiene una cantidad infinita
de nombres de constante que no ocurren en las sentencias de $\Sigma$.
Lo probaremos por el absurdo: supongamos que hay una sentencia $\varphi_0$ tal que
$T \models \varphi_0$ y $T \nvdash \varphi_0$.

Nótese que, ya que $T \nvdash \varphi_0$, tenemos que
$[\varphi_0]_T \ne 1^T = \{\varphi \in S^\tau : T \vdash \varphi\}$
O sea que $[\lnot \varphi_0]_T \ne 0^T$.

Por el Lema de enumeracion hay una infinitupla $(\gamma_1,\gamma_2,\dots)\in F^{\tau \mathbb N}$ tal que:

\begin{itemize}
    \item $|Li(\gamma_j)| \le 1$ para cada $j = 1,2,\dots$
    \item Si $|Li(\gamma)| \le 1$, entonces $\gamma = \gamma_j$ para algún $j \in \mathbb N$.
\end{itemize}

Para cada $j\in\mathbb N$, sea $w_j \in Var$ tal que $Li(\gamma_j) \subseteq \{w_j\}$.
Para cada $j$, declaremos $\gamma_j \; {:=}_d\; \gamma_j(w_j)$.

Nótese que por el Lema del Infimo tenemos que
\[
\inf\bigl(\{[\gamma_j(t)]_T : t \in T^\tau_c\}\bigr)
= [\forall w_j\, \gamma_j(w_j)]_T,
\qquad j=1,2,\dots
\]

Por el Teorema de Rasiowa–Sikorski, existe un filtro primo $U$ de $A_T$ tal que:

\begin{enumerate}[label=(\alph*)]
    \item $[\lnot \varphi_0]_T \in U$,
    \item Para cada $j\in\mathbb N$,
    \[
    \{[\gamma_j(t)]_T : t\in T^\tau_c\} \subseteq U
    \quad\Longrightarrow\quad
    [\forall w_j\, \gamma_j(w_j)]_T \in U.
    \]
\end{enumerate}

Dado que la infinitupla $(\gamma_1,\gamma_2,\dots)$ cubre todas las fórmulas
con a lo sumo una variable libre, podemos reescribir la propiedad (b) como:

\begin{enumerate}[label=(b$'$)]
    \item Para cada $\varphi {:=}_d \varphi(v) \in F^\tau$, si
    \[
    \{[\varphi(t)]_T : t \in T^\tau_c\} \subseteq U,
    \]
    entonces
    \[
    [\forall v\, \varphi(v)]_T \in U.
    \]
\end{enumerate}

Definimos sobre $T^\tau_c$ la siguiente relación:
\[
t \,\bowtie\, s \quad\text{si y sólo si}\quad [(t \equiv s)]_T \in U.
\]

Definamos ahora un modelo \(A_{U}\) de tipo \(\tau\) de la siguiente manera:

\begin{itemize}
    \item Universo de \(A_{U} = T^{\tau}_{c} / \mathrel{\bowtie}\).
    
    \item \(c^{A_{U}} = c / \mathrel{\bowtie}\), para cada \(c \in C\).
    
    \item \(f^{A_{U}}(t_{1}/\mathrel{\bowtie},\dots,t_{n}/\mathrel{\bowtie}) 
    = f(t_{1},\dots,t_{n}) / \mathrel{\bowtie}\), para cada \(f \in F_{n}\) y 
    \(t_{1},\dots,t_{n} \in T^{\tau}_{c}\).
    
    \item \(r^{A_{U}} = \{(t_{1}/\mathrel{\bowtie},\dots,t_{n}/\mathrel{\bowtie}) 
    : [\, r(t_{1},\dots,t_{n}) \,]_{T} \in U\}, \quad \text{para cada } r \in R_{n}.\)
\end{itemize}

Entonces se verifica que:

\begin{enumerate}
    \item $\bowtie$ es una relación de equivalencia.

    \item Para cada $\varphi {:=}_d \varphi(v_1,\dots,v_n)\in F_\tau$ y
    $t_1,\dots,t_n,s_1,\dots,s_n \in T^\tau_c$, si
    \[
    t_1 \bowtie s_1,\; t_2 \bowtie s_2,\;\dots,\; t_n \bowtie s_n,
    \]
    entonces
    \[
    [\varphi(t_1,\dots,t_n)]_T \in U
    \quad\Longleftrightarrow\quad
    [\varphi(s_1,\dots,s_n)]_T \in U.
    \]

    \item Para cada $f \in F_n$ y
    $t_1,\dots,t_n,s_1,\dots,s_n \in T^\tau_c$,
    \[
    t_1 \bowtie s_1,\ t_2 \bowtie s_2,\ \dots,\ t_n \bowtie s_n
    \quad\Longrightarrow\quad
    f(t_1,\dots,t_n)\; \bowtie\; f(s_1,\dots,s_n).
    \]

    \item Para cada $t {:=}_d t(v_1,\dots,v_n) \in T^\tau$ y 
    $t_1,\dots,t_n \in T^\tau_c$, tenemos que
    \[
    t^{A_U}[t_1/\!\bowtie,\dots,t_n/\!\bowtie]
    =
    t(t_1,\dots,t_n)/\!\bowtie.
    \]

    \item Para cada $\varphi {:=}_d \varphi(v_1,\dots,v_n)\in F_\tau$ y 
    $t_1,\dots,t_n \in T^\tau_c$, tenemos que
    \[
    A_U \models \varphi[t_1/\!\bowtie,\dots,t_n/\!\bowtie]
    \quad\text{si y sólo si}\quad
    [\varphi(t_1,\dots,t_n)]_T \in U.
    \]
\end{enumerate}

Probaremos (2). Nótese que
\[
T \;\vdash\; 
\bigl( (t_{1} \equiv s_{1}) \;\wedge\; (t_{2} \equiv s_{2}) \;\wedge\; \cdots \;\wedge\; (t_{n} \equiv s_{n}) 
\;\wedge\; \varphi(t_{1},\dots,t_{n}) \bigr)
\;\to\;
\varphi(s_{1},\dots,s_{n}),
\]
lo cual nos dice que
\[
[(t_{1} \equiv s_{1})]_{T}
\; i^{T} \;
[(t_{2} \equiv s_{2})]_{T}
\; i^{T} \;\cdots\; i^{T} \;
[(t_{n} \equiv s_{n})]_{T}
\; i^{T} \;
[\varphi(t_{1},\dots,t_{n})]_{T}
\;\le^{T}\;
[\varphi(s_{1},\dots,s_{n})]_{T}.
\]

De esto se desprende que
\[
[\varphi(t_{1},\dots,t_{n})]_{T} \in U 
\quad\Longrightarrow\quad
[\varphi(s_{1},\dots,s_{n})]_{T} \in U,
\]
ya que \(U\) es un filtro. La otra implicación es análoga.

Para probar (3) podemos tomar 
\[
\varphi \;=\; \bigl(f(v_{1},\dots,v_{n}) \equiv f(s_{1},\dots,s_{n})\bigr)
\]
y aplicar (2).

\subsubsection*{Prueba del ítem (4).}
\emph{Afirmación.} Para todo término \(t =_{d} t(v_1,\dots,v_n)\in T^\tau\) y para todo
\(t_1,\dots,t_n \in T^\tau_c\) se tiene
\[
t^{A_U}[t_1/\!\bowtie,\dots,t_n/\!\bowtie]
= 
t(t_1,\dots,t_n)/\!\bowtie.
\]

Por inducción sobre la formación del término \(t\).

\textbf{Caso base.}
\begin{itemize}
  \item Si \(t\) es la variable \(v_i\), entonces por definición de la evaluación en \(A_U\)
  \[
  t^{A_U}[t_1/\!\bowtie,\dots,t_n/\!\bowtie]
  \text{ (Carácter recursivo de la notación $t^A[a_1,...,a_n]$)}
   \]
   \[
  = t_i/\!\bowtie \text{ (Def $A_U$)}
   \]
   \[
  = t(t_1,\dots,t_n)/\!\bowtie \text{ (Notacion Declaratoria)}
  \]
  \item Si \(t\) es una constante \(c\)   \[
  t^{A_U}[t_1/\!\bowtie,\dots,t_n/\!\bowtie] = c^{A_U} \text{ (Carácter recursivo de la notación $t^A[a_1,...,a_n]$)}
  \]
  \[ 
  = c/\!\bowtie,\text{ (Def $A_U$)}
  \]
  \[
  = t(t_1,\dots,t_n)/\!\bowtie \text{ (Notacion Declaratoria)}
  \]
\end{itemize}

\textbf{Paso inductivo.} Supongamos que la afirmación vale para todos los subtérminos de complejidad menor que \(t\).  
Sea \(t\) de la forma \(f(s_1,\dots,s_m)\), donde \(f\) es un símbolo de función \(m\)-ario y \(s_1,\dots,s_m\) son términos más simples. Entonces, por la definición de evaluación en la estructura \(A_U\),
\[
t^{A_U}[t_1/\!\bowtie,\dots,t_n/\!\bowtie]
= f^{A_U}\bigl(s_1^{A_U}[t_1/\!\bowtie,\dots,t_n/\!\bowtie],\dots,
s_m^{A_U}[t_1/\!\bowtie,\dots,t_n/\!\bowtie]\bigr).
\]
Aplicando la hipótesis inductiva a cada \(s_j\) obtenemos
\[
s_j^{A_U}[t_1/\!\bowtie,\dots,t_n/\!\bowtie] = s_j(t_1,\dots,t_n)/\!\bowtie,
\qquad j=1,\dots,m.
\]
Por definición de la interpretación \(f^{A_U}\) en \(A_U\) (la cual toma clases de términos y devuelve la clase del término formado por \(f\)),
\[
f^{A_U}\bigl(s_1(t_1,\dots,t_n)/\!\bowtie,\dots,s_m(t_1,\dots,t_n)/\!\bowtie\bigr)
= f(s_1(t_1,\dots,t_n),\dots,s_m(t_1,\dots,t_n))/\!\bowtie.
\]
Pero el miembro derecho es exactamente \(t(t_1,\dots,t_n)/\!\bowtie\), puesto que \(t=f(s_1,\dots,s_m)\).  
De esta forma concluimos
\[
t^{A_U}[t_1/\!\bowtie,\dots,t_n/\!\bowtie] = t(t_1,\dots,t_n)/\!\bowtie,
\]
y la inductiva queda probada.

\(\square\)

\end{proof}

\section*{Auxiliares de Demostraciones}

\begin{enumerate}
    \item \textbf{(Lema de enumeración).} Sea $\tau$ un tipo.  
    Hay una infinitupla $(\gamma_1,\gamma_2,\ldots)\in F^{\tau^{\mathbb{N}}}$ tal que:

    \begin{enumerate}
        \item $\lvert L_i(\gamma_j)\rvert \le 1$, para cada $j = 1,2,\ldots$
        \item Si $\lvert L_i(\gamma)\rvert \le 1$, entonces $\gamma = \gamma_j$ para algún $j\in\mathbb{N}$.
    \end{enumerate}

    \item \textbf{(Lema del Ínfimo).} Sea $T = (\Sigma, \tau)$ una teoría y supongamos que $\tau$ tiene una cantidad infinita 
    de nombres de constante que no ocurren en las sentencias de $\Sigma$. 
    Entonces, para cada fórmula $\varphi =_{d} \varphi(v)$, se tiene que en el álgebra de Lindenbaum $A_T$:

    \[
    [\forall v\, \varphi(v)]_T 
    = 
    \inf\bigl(\{\, [\varphi(t)]_T : t \in T_c^\tau \,\}\bigr).
    \]

    \item \textbf{(Teorema de Rasiowa y Sikorski).} 
    Sea $(B, s, i, c, 0, 1)$ un álgebra de Boole. 
    Sea $a \in B$, $a \neq 0$. 
    Supongamos que $(A_1, A_2, \ldots)$ es una infinitupla de subconjuntos de $B$ tal que existe 
    $\inf(A_j)$ para cada $j = 1,2,\ldots$. 
    Entonces existe un filtro primo $P$ que cumple:

    \begin{enumerate}
        \item[(a)] $a \in P$,
        \item[(b)] Si $P \supseteq A_j$, entonces $\,\inf(A_j) \in P$, para cada $j = 1,2,\ldots$.
    \end{enumerate}


\end{enumerate}

\end{document}
