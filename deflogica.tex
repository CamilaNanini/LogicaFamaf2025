\documentclass[12pt]{article}
\usepackage[utf8]{inputenc}
\usepackage[spanish]{babel}
\usepackage{enumitem}
\usepackage{amsmath, amssymb}

\title{Combos de Definiciones}
\author{Camila Nanini}

\begin{document}

\maketitle

\section*{Combo 1}
\begin{enumerate}[label=(\arabic*)]
    \item Defina $n(J)$ (para $J \in \text{Just}^+$)

    \bigskip
    Sea $J \in \text{Just}^+$. Por el Lema 4, existen únicos $n \geq 1$ y $J_1, \ldots, J_n \in \text{Just}$ tales que 
    \[
    J = J_1 \ldots J_n.
    \]
    Definimos entonces $n(J) = n$, y escribiremos $J_1, \ldots, J_{n(J)}$ para referirnos a la única sucesión de justificaciones cuya concatenación es $J$.

    \item Defina “par adecuado de tipo $\tau$” (no hace falta que defina cuando $J \in \text{Just}^+$ es balanceada)

    \bigskip
    Un \textbf{par adecuado de tipo} $\tau$ es un par $(\varphi, J) \in S^{ \tau +} \times \text{Just}^+$ tal que se cumple que $n(\varphi) = n(J)$ y $J$ es balanceada.  

    \item Defina $\text{Mod}_T(\varphi)$
    \medskip
    
    Sea $T = (\Sigma, \tau)$ una teoría. Dada $\varphi \in S_{\tau}$ definamos
    \[
    \mathrm{Mod}_T(\varphi) = \{A : A \text{ es modelo de } T \text{ y } A \vDash \varphi\}.
    \]

    \item Dados $\varphi =_d \varphi(v_1, \ldots, v_n)$, $A$ una estructura de tipo $\tau$ y $a_1, \ldots, a_n \in A$, defina qué significa $A \vDash \varphi[a_1, \ldots, a_n]$ (i.e. Convención notacional 4)
    \bigskip
    Sean 
    \[\varphi =_d \varphi(v_1, \dots, v_n)\]
    una fórmula de tipo $\tau$, 
    $A$ una estructura (modelo) de tipo $\tau$, 
    y $a_1, \dots, a_n \in A$. 

    Entonces escribimos
    \[
        A \vDash \varphi[a_1, \dots, a_n]
    \]
    para indicar que existe una asignación 
    \(\vec{b}\) tal que a cada variable \(v_i\) le asigna el valor \(a_i\),
    y se cumple que 
    \[
        A \vDash \varphi[\vec{b}],
    \]
    es decir, la fórmula \(\varphi\) resulta verdadera en \(A\) bajo dicha asignación.  

    \item Defina $(L,s,i,c,0,1)/\theta$ ($\theta$ una congruencia del reticulado complementado $(L,s,i,c,0,1)$).  
    
    \bigskip
    
    Sea $(L,s,i,c,0,1)$ un reticulado complementado. Una congruencia sobre $(L,s,i,c,0,1)$ es una relación de equivalencia $\theta$ sobre $L$ tal que:
    \begin{enumerate}[label=(\roman*)]
        \item $\theta$ es una congruencia sobre $(L,s,i,0,1)$
        \item $x/\theta = y/\theta \implies x^c/\theta = y^c/\theta$
    \end{enumerate}
    
    Las condiciones anteriores nos permiten definir sobre $L/\theta$ dos operaciones binarias $\tilde{s}$ y $\tilde{\imath}$, y una operación unaria $\tilde{c}$ de la siguiente manera:
    \[
        x/\theta \;\tilde{s}\; y/\theta = (x \;s\; y)/\theta, \quad
        x/\theta \;\tilde{\imath}\; y/\theta = (x \;i\; y)/\theta, \quad
        (x/\theta)^{\tilde{c}} = x^c/\theta
    \]
    
    La 6-upla $(L/\theta, \tilde{s}, \tilde{\imath}, \tilde{c}, 0/\theta, 1/\theta)$ se llama el cociente de $(L,s,i,c,0,1)$ sobre $\theta$ y se denota por $(L,s,i,c,0,1)/\theta$.
\end{enumerate}


\section*{Combo 2}
\begin{enumerate}[label=(\arabic*)]
    \item Defina $(\Sigma, \tau) \vDash \varphi$

    \bigskip
    Sea $(\Sigma, \tau)$ una teoría de primer orden. Decimos que
    \[
    (\Sigma, \tau) \vDash \varphi
    \]
    si y sólo si $\varphi$ es verdadera en todo modelo de $(\Sigma, \tau)$.

    \item Defina “Partición de $A$” y $RP$
    
    Dado un conjunto $A$, por una \emph{partición} de $A$ entenderemos un conjunto $P$ tal que:
    \begin{itemize}
    \item Cada elemento de $P$ es un subconjunto no vacío de $A$.
    \item Si $S_1, S_2 \in P$ y $S_1 \neq S_2$, entonces $S_1 \cap S_2 = \emptyset$.
    \item $A = \{ a : a \in S \text{ para algún } S \in P \}$.
    \end{itemize}

    Dada una partición $P$ de un conjunto $A$, podemos definir una relación binaria asociada a $P$ de la siguiente manera:
    \[
    R_P = \{ (a, b) \in A^2 : a, b \in S \text{ para algún } S \in P \}.
    \]

    \item Defina cuando “$\varphi_i$ está bajo la hipótesis $\varphi_l$ en $(\varphi, J)$” (no hace falta que defina $B^J$)
    
    Diremos que $\varphi_i$ está bajo la hipótesis $\varphi_l$ en $(\varphi, J)$, o que 
    $\varphi_l$ es una hipótesis de $\varphi_i$ en $(\varphi, J)$, cuando haya en $B^J$ 
    un bloque de la forma $\langle l, j \rangle$ el cual contenga a $i$.

    \item Defina $(L,s,i)/\theta$ ($\theta$ una congruencia del reticulado terna $(L,s,i)$).  
    
    \bigskip

    Sea $(L,s,i)$ un reticulado terna. Una congruencia sobre $(L,s,i)$ es una relación de equivalencia $\theta$ sobre $L$ tal que:
    \[
        x \,\theta\, x' \;\text{y}\; y \,\theta\, y' \implies (x \,s\, y) \,\theta\, (x' \,s\, y') \;\text{y}\; (x \,i\, y) \,\theta\, (x' \,i\, y')
    \]
    
    Gracias a esta condición podemos definir de manera inambigua sobre $L/\theta$ dos operaciones binarias $\tilde{s}$ y $\tilde{\imath}$ de la siguiente manera:
    \[
        x/\theta \;\tilde{s}\; y/\theta = (x \,s\, y)/\theta, \quad
        x/\theta \;\tilde{\imath}\; y/\theta = (x \,i\, y)/\theta
    \]
\end{enumerate}


\section*{Combo 3}
\begin{enumerate}[label=(\arabic*)]
    \item Dados $t \stackrel{d}{=} t(v_1, \ldots, v_n) \in T^\tau$, $A$ una estructura de tipo $\tau$ y $a_1, \ldots, a_n \in A$, defina $t^A[a_1, \ldots, a_n]$ (i.e. Convención notacional 2)

    \bigskip
    
    Dados $t \stackrel{d}{=} t(v_1, \ldots, v_n) \in T^\tau$, $A$ una estructura de tipo $\tau$ y $a_1, \ldots, a_n \in A$, se define:
    \[
    t^A[a_1, \ldots, a_n]
    \]
    como el elemento $t^A[\vec{b}]$, donde $\vec{b}$ es una asignación tal que a cada variable $v_i$ le asigna el valor $a_i$.  

    \item Defina “$F$ es un homomorfismo de $(L,s,i,c,0,1)$ en $(L',s',i',c',0',1')$”.  
    
    \bigskip
    
    Sean $(L,s,i,c,0,1)$ y $(L',s',i',c',0',1')$ reticulados complementados. Una función $F : L \to L'$ se llama homomorfismo de $(L,s,i,c,0,1)$ en $(L',s',i',c',0',1')$ si para todo $x, y \in L$ se cumple:
    \[
        F(x \,s\, y) = F(x) \,s'\, F(y), \quad
        F(x \,i\, y) = F(x) \,i'\, F(y),
    \]
    \[
        F(x^c) = F(x)^{c'}, \quad
        F(0) = 0', \quad
        F(1) = 1'.
    \]
    
    \item Defina “filtro generado por $S$ en $(L,s,i)$”
    
    Dado un conjunto $S \subseteq L$, denotemos con $[S)$ el siguiente conjunto:
    \[
    [S) = \{\, y \in L : y \geq s_1 \lor \dots \lor s_n, \text{ para algunos } s_1, \ldots, s_n \in S, \, n \geq 1 \,\}.
    \]
    Llamaremos a $[S)$ el \emph{filtro generado por} $S$.

    \item Defina cuando $J \in \text{Just}^+$ es balanceada (no hace falta que defina $B_J$)
    \bigskip

Sea $J = J_1 \dots J_{n(J)} \in \text{Just}^+$. Diremos que $J$ es \emph{balanceada} si se cumplen las siguientes condiciones:

\begin{enumerate}
    \item Para cada $k \in \mathbb{N}$, a lo sumo hay un $i$ tal que $J_i = \text{HIPOTESIS } \overline{k}$ y a lo sumo hay un $i$ tal que $J_i = \text{TESIS } \overline{k}\alpha$, con $\alpha \in \text{JustBas}$.
    \item Si $J_i = \text{HIPOTESIS } \overline{k}$, entonces existe $l > i$ tal que $J_l = \text{TESIS } \overline{k}\alpha$, con $\alpha \in \text{JustBas}$.
    \item Si $J_i = \text{TESIS } \overline{k}\alpha$, con $\alpha \in \text{JustBas}$, entonces existe $l < i$ tal que $J_l = \text{HIPOTESIS } \overline{k}$.
    \item Los bloques de hipótesis-tesis no se solapan de forma parcial: si $B_1, B_2 \in B^J$, entonces $B_1 \cap B_2 = \emptyset$ o $B_1 \subseteq B_2$ o $B_2 \subseteq B_1$.
\end{enumerate}

\end{enumerate}

\section*{Combo 4} 
\begin{enumerate}[label=(\arabic*)]
    \item Defina “$(L,s,i,c,0,1)$ es un subreticulado complementado de $(L',s',i',c',0',1')$”.  
    
    \bigskip

    Dados reticulados complementados $(L,s,i,c,0,1)$ y $(L',s',i',c',0',1')$, diremos que $(L,s,i,c,0,1)$ es un subreticulado complementado de $(L',s',i',c',0',1')$ si se cumplen las siguientes condiciones:
    \begin{enumerate}[label=(\roman*)]
        \item $L \subseteq L'$
        \item $L$ es cerrado bajo las operaciones $s'$, $i'$ y $c'$
        \item $0 = 0'$ y $1 = 1'$
        \item $s = s'|_{L \times L}, \; i = i'|_{L \times L}, \; c = c'|_L$
    \end{enumerate}
    
    \item Defina $A \vDash \varphi[\vec{a}]$ (versión absoluta, no dependiente de una declaración previa, i.e. $\vec{a} \in A^N$. No hace falta definir $t^A[\vec{a}]$)
    
    \bigskip

    Sea $A = (A, i)$ una estructura de tipo $\tau$, y sea $\vec{a} = (a_1, a_2, \ldots) \in A^N$ una asignación.  
    Definimos recursivamente la relación de \emph{satisfacción} $A \vDash \varphi[\vec{a}]$ como:

    \begin{enumerate}[label=(\arabic*)]
        \item Si $\varphi$ es una \textbf{fórmula atómica}:
        \begin{itemize}
            \item Si $\varphi$ es de la forma $t = s$, entonces
            \[
            A \vDash (t = s)[\vec{a}] \;\; \text{ssi} \;\; t^A[\vec{a}] = s^A[\vec{a}].
            \]
            \item Si $\varphi$ es de la forma $r(t_1, \ldots, t_n)$, con $r \in R_n$, entonces
            \[
            A \vDash r(t_1, \ldots, t_n)[\vec{a}] \;\; \text{ssi} \;\; (t_1^A[\vec{a}], \ldots, t_n^A[\vec{a}]) \in r^A.
            \]
        \end{itemize}

        \item Si $\varphi = \neg \psi$, entonces
        \[
        A \vDash \neg \psi[\vec{a}] \;\; \text{ssi} \;\; A \nvDash \psi[\vec{a}].
        \]
    
        \item Si $\varphi = (\psi_1 \eta \psi_2)$, donde $\eta \in \{\land, \lor, \to, \leftrightarrow\}$, entonces
        \[
        A \vDash (\psi_1 \eta \psi_2)[\vec{a}]
        \]
        se define:
        \begin{align*}
            A \vDash (\psi_1 \land \psi_2)[\vec{a}] &\;\; \text{ssi} \;\; A \vDash \psi_1[\vec{a}] \text{ y } A \vDash \psi_2[\vec{a}],\\
            A \vDash (\psi_1 \lor \psi_2)[\vec{a}] &\;\; \text{ssi} \;\; A \vDash \psi_1[\vec{a}] \text{ o } A \vDash \psi_2[\vec{a}],\\
            A \vDash (\psi_1 \to \psi_2)[\vec{a}] &\;\; \text{ssi} \;\; A \nvDash \psi_1[\vec{a}] \text{ o } A \vDash \psi_2[\vec{a}],\\
            A \vDash (\psi_1 \leftrightarrow \psi_2)[\vec{a}] &\;\; \text{ssi} \;\; 
            (A \vDash \psi_1[\vec{a}] \text{ y } A \vDash \psi_2[\vec{a}]) \text{ o } 
            (A \nvDash \psi_1[\vec{a}] \text{ y } A \nvDash \psi_2[\vec{a}]).
        \end{align*}

        \item Si $\varphi = Qx_i\,\psi$, con $Q \in \{\forall, \exists\}$, entonces:
        \begin{itemize}
            \item $A \vDash \forall x_i\,\psi[\vec{a}]$ ssi para todo $a \in A$, se tiene $A \vDash \psi[\downarrow_i^a\vec{a}]$.
            \item $A \vDash \exists x_i\,\psi[\vec{a}]$ ssi existe $a \in A$ tal que $A \vDash \psi[\downarrow_i^a\vec{a}]$.
        \end{itemize}
    \end{enumerate}

    En tal caso decimos que \emph{$A$ satisface $\varphi$ bajo la asignación $\vec{a}$}, o que \emph{$\varphi$ es verdadera en $A$ para $\vec{a}$}.


    \item Defina la relación “$v$ ocurre libremente en $\varphi$ a partir de $i$”
    
    Definimos recursivamente la relación “$v$ ocurre libremente en $\varphi$ a partir de $i$”, donde $v \in \text{Var}$, $\varphi \in F_\tau$ e $i \in \{1, \dots, |\varphi|\}$, de la siguiente manera:

    \begin{enumerate}[label=(\arabic*)]
        \item Si $\varphi$ es atómica, entonces $v$ ocurre libremente en $\varphi$ a partir de $i$ si y solo si $v$ ocurre en $\varphi$ a partir de $i$.
        \item Si $\varphi = (\varphi_1 \eta \varphi_2)$, entonces $v$ ocurre libremente en $\varphi$ a partir de $i$ si y solo si se da alguna de las siguientes:
        \begin{enumerate}[label=(\alph*)]
            \item $v$ ocurre libremente en $\varphi_1$ a partir de $i-1$
            \item $v$ ocurre libremente en $\varphi_2$ a partir de $i - |(\varphi_1 \eta|$
        \end{enumerate}
        \item Si $\varphi = \neg \varphi_1$, entonces $v$ ocurre libremente en $\varphi$ a partir de $i$ si y solo si $v$ ocurre libremente en $\varphi_1$ a partir de $i-1$.
        \item Si $\varphi = Q w \varphi_1$, entonces $v$ ocurre libremente en $\varphi$ a partir de $i$ si y solo si $v \neq w$ y $v$ ocurre libremente en $\varphi_1$ a partir de $i - |Q w|$.
    \end{enumerate}


    \item Defina reticulado cuaterna.  
    
    
    Por un reticulado cuaterna entendemos una 4-upla $(L,s,i,\leq)$ tal que $L$ es un conjunto no vacío, $s$ e $i$ son operaciones binarias sobre $L$, $\leq$ es una relación binaria sobre $L$, y se cumplen las siguientes propiedades:
    \begin{enumerate}[label=(\roman*)]
        \item $x \leq x$, cualesquiera sea $x \in L$
        \item $x \leq y$ y $y \leq z$ implican $x \leq z$, cualesquiera sean $x, y, z \in L$
        \item $x \leq y$ y $y \leq x$ implican $x = y$, cualesquiera sean $x, y \in L$
        \item $x \leq x \,s\, y$ y $y \leq x \,s\, y$, cualesquiera sean $x, y \in L$
        \item $x \leq z$ y $y \leq z$ implican $x \,s\, y \leq z$, cualesquiera sean $x, y, z \in L$
        \item $x \,i\, y \leq x$ y $x \,i\, y \leq y$, cualesquiera sean $x, y \in L$
        \item $z \leq x$ y $z \leq y$ implican $z \leq x \,i\, y$, cualesquiera sean $x, y, z \in L$
    \end{enumerate}
\end{enumerate}

\section*{Combo 5}
Explique la notación declaratoria para términos con sus tres Convenciones Notacionales (Convenciones 1, 2 y 5 de la Guía 11).

\bigskip

    La notación declaratoria para términos se utiliza para expresar explícitamente de qué variables depende un término y permite realizar sustituciones o evaluaciones.  
    Supongamos que $t =_d t(v_1, \ldots, v_n)$ es una declaración de término. Entonces se aplican las siguientes convenciones:

    \begin{enumerate}
    \item \textbf{Convención Notacional 1:}  
      Si hemos declarado $t =_d t(v_1, \ldots, v_n)$ y $P_1, \ldots, P_n$ son palabras cualesquiera (no necesariamente términos), entonces:
    \[
    t(P_1, \ldots, P_n)
    \]
    denota la palabra que resulta de reemplazar simultáneamente cada ocurrencia de $v_1$ en $t$ por $P_1$, cada ocurrencia de $v_2$ por $P_2$, y así sucesivamente.  

    \item \textbf{Convención Notacional 2:}  
    Si hemos declarado $t =_d t(v_1, \ldots, v_n)$, $A$ es un modelo de tipo $\tau$ y $a_1, \ldots, a_n \in A$, entonces:
    \[
    t^A[a_1, \ldots, a_n]
    \]
    denota el elemento $t^A[\vec{b}]$, donde $\vec{b}$ es una asignación tal que a cada variable $v_i$ le asigna el valor $a_i$.  

    \item \textbf{Convención Notacional 5:}  
    Si hemos declarado $t =_d t(v_1, \ldots, v_n)$ y se cumple el caso (3) del Lema de Lectura Única de Términos —es decir, 
    \[
    t = f(t_1, \ldots, t_m),
    \]
    con $f \in F_m$ y $t_1, \ldots, t_m \in T^\tau$—, entonces supondremos tácitamente que también se han hecho las declaraciones:
    \[
    t_1 =_d t_1(v_1, \ldots, v_n), \quad \ldots, \quad t_m =_d t_m(v_1, \ldots, v_n).
    \]
    Esto es posible porque las variables que ocurren en cada $t_i$ pertenecen al conjunto $\{v_1, \ldots, v_n\}$.  
    \end{enumerate}

\section*{Combo 6}
Explique la notación declaratoria para fórmulas con sus tres Convenciones Notacionales (Convenciones 3, 4 y 6 de la Guía 11). Puede asumir la notación declaratoria para términos.

\bigskip

Supongamos que hemos declarado una fórmula $\varphi =_d \varphi(v_1, \ldots, v_n)$. Entonces se aplican las siguientes convenciones:

\begin{enumerate}
  \item \textbf{Convención Notacional 3:}  
  Si hemos hecho la declaración $\varphi =_d \varphi(v_1, \ldots, v_n)$ y $P_1, \ldots, P_n$ son palabras cualesquiera (no necesariamente términos o fórmulas), entonces:
  \[
  \varphi(P_1, \ldots, P_n)
  \]
  denota la palabra que resulta de reemplazar simultáneamente cada ocurrencia \emph{libre} de $v_1$ en $\varphi$ por $P_1$, cada ocurrencia libre de $v_2$ por $P_2$, y así sucesivamente.  

  \item \textbf{Convención Notacional 4:}  
  Si hemos declarado $\varphi =_d \varphi(v_1, \ldots, v_n)$, y $A$ es una estructura (modelo) de tipo $\tau$ con $a_1, \ldots, a_n \in A$, entonces:
  \[
  A \vDash \varphi[a_1, \ldots, a_n]
  \]
  significa que $A \vDash \varphi[\vec{b}]$, donde $\vec{b}$ es una asignación tal que a cada variable $v_i$ le asigna el valor $a_i$.  
  En general, $A \nvDash \varphi[a_1, \ldots, a_n]$ significa que no se cumple $A \vDash \varphi[a_1, \ldots, a_n]$.  

  \item \textbf{Convención Notacional 6:}  
  Si hemos declarado $\varphi =_d \varphi(v_1, \ldots, v_n)$, entonces, según la \textbf{Lectura única de fórmulas declaradas (Lema 3)}, se cumple exactamente uno de los siguientes casos para $\varphi$, y en cada uno de ellos se asumen tácitamente las declaraciones indicadas:

  \begin{itemize}
    \item \textbf{(1)} Si $\varphi = (t \equiv s)$, con $t, s \in T^\tau$, entonces también declaramos:
    \[
    t =_d t(v_1, \ldots, v_n) \quad \text{y} \quad s =_d s(v_1, \ldots, v_n).
    \]

    \item \textbf{(2)} Si $\varphi = r(t_1, \ldots, t_m)$, con $r \in R_m$ y $t_1, \ldots, t_m \in T^\tau$, entonces declaramos:
    \[
    t_1 =_d t_1(v_1, \ldots, v_n), \ldots, t_m =_d t_m(v_1, \ldots, v_n).
    \]

    \item \textbf{(3)} Si $\varphi = (\varphi_1 \wedge \varphi_2)$, entonces declaramos:
    \[
    \varphi_1 =_d \varphi_1(v_1, \ldots, v_n), \quad \varphi_2 =_d \varphi_2(v_1, \ldots, v_n).
    \]

    \item \textbf{(4)} Si $\varphi = (\varphi_1 \vee \varphi_2)$, se declaran igualmente:
    \[
    \varphi_1 =_d \varphi_1(v_1, \ldots, v_n), \quad \varphi_2 =_d \varphi_2(v_1, \ldots, v_n).
    \]

    \item \textbf{(5)} Si $\varphi = (\varphi_1 \to \varphi_2)$, se declaran:
    \[
    \varphi_1 =_d \varphi_1(v_1, \ldots, v_n), \quad \varphi_2 =_d \varphi_2(v_1, \ldots, v_n).
    \]

    \item \textbf{(6)} Si $\varphi = (\varphi_1 \leftrightarrow \varphi_2)$, se declaran:
    \[
    \varphi_1 =_d \varphi_1(v_1, \ldots, v_n), \quad \varphi_2 =_d \varphi_2(v_1, \ldots, v_n).
    \]

    \item \textbf{(7)} Si $\varphi = \neg \varphi_1$, se declara:
    \[
    \varphi_1 =_d \varphi_1(v_1, \ldots, v_n).
    \]

    \item \textbf{(8)} Si $\varphi = \forall v_j \, \varphi_1$, con $v_j \in \{v_1, \ldots, v_n\}$, se declara:
    \[
    \varphi_1 =_d \varphi_1(v_1, \ldots, v_n).
    \]

    \item \textbf{(9)} Si $\varphi = \forall v \, \varphi_1$, con $v \in Var - \{v_1, \ldots, v_n\}$, se declara:
    \[
    \varphi_1 =_d \varphi_1(v_1, \ldots, v_n, v).
    \]

    \item \textbf{(10)} Si $\varphi = \exists v_j \, \varphi_1$, con $v_j \in \{v_1, \ldots, v_n\}$, se declara:
    \[
    \varphi_1 =_d \varphi_1(v_1, \ldots, v_n).
    \]

    \item \textbf{(11)} Si $\varphi = \exists v \, \varphi_1$, con $v \in Var - \{v_1, \ldots, v_n\}$, se declara:
    \[
    \varphi_1 =_d \varphi_1(v_1, \ldots, v_n, v).
    \]
  \end{itemize}

\end{enumerate}

\section*{Combo 7}
\begin{enumerate}[label=(\arabic*)]
    \item Defina recursivamente la relación “$v$ es sustituible por $w$ en $\varphi$”
    
    Diremos que una variable $v$ es \emph{sustituible por} $w$ en una fórmula $\varphi$ cuando
    ninguna ocurrencia libre de $v$ en $\varphi$ aparece dentro de una subfórmula de la forma
    $Qw\psi$ (donde $Q$ es un cuantificador, $\forall$ o $\exists$).  
    En otras palabras, $v$ \emph{no} es sustituible por $w$ en $\varphi$ cuando alguna ocurrencia
    libre de $v$ en $\varphi$ ocurre dentro de una subfórmula de la forma $Qw\psi$.

    A partir de esta idea, la relación “$v$ es sustituible por $w$ en $\varphi$” puede definirse
    \textbf{recursivamente} de la siguiente manera:

    \begin{enumerate}
    \item[(1)] Si $\varphi$ es \textbf{atómica}, entonces $v$ es sustituible por $w$ en $\varphi$.
  
    \item[(2)] Si $\varphi = (\varphi_1 \, \eta \, \varphi_2)$, donde $\eta$ es un conector binario 
    ($\wedge$, $\vee$, $\rightarrow$, $\leftrightarrow$), entonces:
    \[
    v \text{ es sustituible por } w \text{ en } \varphi 
    \iff 
    v \text{ es sustituible por } w \text{ en } \varphi_1 
    \text{ y en } \varphi_2.
    \]
  
    \item[(3)] Si $\varphi = \neg \varphi_1$, entonces:
    \[
    v \text{ es sustituible por } w \text{ en } \varphi 
    \iff 
    v \text{ es sustituible por } w \text{ en } \varphi_1.
    \]
  
    \item[(4)] Si $\varphi = Qv \, \varphi_1$, entonces $v$ es sustituible por $w$ en $\varphi$.
  
    \item[(5)] Si $\varphi = Qw \, \varphi_1$ y $v \in Li(\varphi_1)$, entonces 
    $v$ \textbf{no} es sustituible por $w$ en $\varphi$.

    \item[(6)] Si $\varphi = Qw \, \varphi_1$ y $v \notin Li(\varphi_1)$, entonces 
    $v$ es sustituible por $w$ en $\varphi$.
  
    \item[(7)] Si $\varphi = Qu \, \varphi_1$, con $u \neq v, w$, entonces:
    \[
    v \text{ es sustituible por } w \text{ en } \varphi 
    \iff 
    v \text{ es sustituible por } w \text{ en } \varphi_1.
    \]
    \end{enumerate}

    Además, dado un término $t$, se dice que $v$ es \emph{sustituible por $t$ en $\varphi$} cuando
    $v$ es sustituible por cada variable que ocurre en $t$ dentro de $\varphi$.

    \item Defina cuando $J \in \text{Just}^+$ es balanceada (no hace falta que defina $B^J$) (Mismo que en combo 3.4)
    \item Defina “filtro del reticulado terna $(L,s,i)$”
    
    Un \emph{filtro} de un reticulado $\,(L, s, i)\,$ será un subconjunto $F \subseteq L$ tal que:
    \begin{enumerate}
        \item $F \neq \emptyset$,
        \item Si $x, y \in F$, entonces $x i y \in F$,
        \item Si $x \in F$ y $x \leq y$, entonces $y \in F$.
    \end{enumerate}

    \item Defina “teoría elemental”.  
    
    Una teoría elemental es un par $(\Sigma, \tau)$ tal que $\tau$ es un tipo cualquiera y $\Sigma$ es un conjunto de sentencias elementales puras de tipo $\tau$.
\end{enumerate}

\section*{Combo 8}
\begin{enumerate}[label=(\arabic*)]
    \item Defina $(L,s,i,c,0,1)/\theta$ ($\theta$ una congruencia del reticulado complementado $(L,s,i,c,0,1)$)
    (Ver combo 1)
    \item Dados $\varphi =_d \varphi(v_1, \ldots, v_n)$, $A$ una estructura de tipo $\tau$ y $a_1, \ldots, a_n \in A$, defina qué significa $A \vDash \varphi[a_1, \ldots, a_n]$ (i.e. Convención notacional 4)

    \bigskip

    Según la Convención Notacional 4, cuando hemos declarado $\varphi =_d \varphi(v_1, \ldots, v_n)$,  
    si $A$ es un modelo de tipo $\tau$ y $a_1, \ldots, a_n \in A$, entonces
    \[
    A \vDash \varphi[a_1, \ldots, a_n]
    \]

    significa que $A \vDash \varphi[\vec{b}]$, donde $\vec{b}$ es una asignación tal que a cada variable $v_i$ le asigna el valor $a_i$.  
    En general, $A \not\vDash \varphi[a_1, \ldots, a_n]$ significa que no sucede $A \vDash \varphi[a_1, \ldots, a_n]$.

    \item Dado un poset $(P, \leq)$, defina “$a$ es supremo de $S$ en $(P, \leq)$”
    
    \bigskip

    Sea $(P, \leq)$ un poset y $S \subseteq P$. 
    Un elemento $a \in P$ se llama supremo de $S$ en $(P, \leq)$ si se cumplen las siguientes propiedades:
    \begin{enumerate}[label=(\arabic*)]
        \item $a$ es cota superior de $S$ en $(P, \leq)$
        \item Para cada $b \in P$, si $b$ es cota superior de $S$ en $(P, \leq)$, entonces $a \leq b$
    \end{enumerate}
    Diremos que un elemento $a \in P$ es cota superior de $S$ en $(P, \leq)$ cuando $b \leq a$ para todo $b \in S$. 

    \item Defina “$i$ es anterior a $j$ en $(\varphi, J)$” (no hace falta que defina $B_J$)

    \bigskip
    Sea $(\varphi, J)$ un par adecuado de tipo $\tau$, con $i, j \in \langle 1, n(\varphi)\rangle$.  
    Diremos que $i$ es \emph{anterior} a $j$ en $(\varphi, J)$ si se cumplen las siguientes condiciones:

    \begin{enumerate}
        \item $i < j$.
        \item Para todo $B \in B^J$, si $i \in B$, entonces $j \in B$.
    \end{enumerate}
\end{enumerate}

\section*{Combo 9}
\begin{enumerate}[label=(\arabic*)]
    \item Defina “término elemental de tipo $\tau$”
    
    
    Dado un tipo $\tau = (C, F, R, a)$, los términos elementales de tipo $\tau$ se definen mediante las siguientes cláusulas:
    \begin{enumerate}[label=(\arabic*)]
        \item Cada palabra de $C$ es un término elemental de tipo $\tau$
        \item Las variables $x, y, z, w, \dots$ son términos elementales de tipo $\tau$
        \item Los nombres de elementos fijos $a, b, c, d, \dots$ son términos elementales de tipo $\tau$
        \item Si $f \in F_n$, con $n \ge 1$, y $t_1, \dots, t_n$ son términos elementales de tipo $\tau$, entonces
        \[
            f(t_1, \dots, t_n)
        \]
        es un término elemental de tipo $\tau$
        \item Una palabra es un término elemental de tipo $\tau$ si y solo si se puede construir usando las cláusulas anteriores
    \end{enumerate}

    \item Defina $\dashv\vdash_T$:

    Sea $T = (\Sigma, \tau)$ una teoría. Definimos la siguiente relación binaria sobre el conjunto de sentencias $S_\tau$:
    \[
    \varphi \dashv\vdash_T \psi \quad \text{si y sólo si} \quad T \vdash (\varphi \leftrightarrow \psi)
    \]
    Es decir,
    \[
    \dashv\vdash_T = \{(\varphi, \psi) \in S_\tau : T \vdash (\varphi \leftrightarrow \psi)\}.
    \]
    La relación $\dashv\vdash_T$ es una relación de equivalencia sobre $S_\tau$.

    \item Defina $s^T$ (explique por qué la definición es inambigua):

    Dada una teoría $T = (\Sigma, \tau)$ y $\varphi, \psi \in S_\tau$, definimos la operación binaria
    \[
    [\varphi]_T \, s^T \, [\psi]_T = [(\varphi \lor \psi)]_T.
    \]
    La definición es inambigua porque si $[\varphi]_T = [\varphi']_T$ y $[\psi]_T = [\psi']_T$, entonces 
    \[
    T \vdash ((\varphi \lor \psi) \leftrightarrow (\varphi' \lor \psi')),
    \]
    lo cual garantiza que las clases de equivalencia obtenidas son las mismas, independientemente de los representantes elegidos.

    \item Defina $A_T$:

    Dada una teoría $T = (\Sigma, \tau)$, se define el \emph{álgebra de Lindenbaum} de $T$ como
    \[
    A_T = (S_\tau / \dashv\vdash_T, \, s^T, \, i^T, \, c^T, \, 0_T, \, 1_T),
    \]
    donde:
    \[
    [\varphi]_T \, i^T \, [\psi]_T = [(\varphi \land \psi)]_T, \quad 
    ([\varphi]_T)^{c^T} = [\neg \varphi]_T,
    \]
    \[
    1_T = \{\varphi \in S_\tau : T \vdash \varphi\}, \quad 
    0_T = \{\varphi \in S_\tau : T \vdash \neg \varphi\}.
    \]
    El álgebra $A_T$ es un álgebra de Boole.

    \item Defina “$S$ es un subuniverso del reticulado complementado $(L,s,i,c,0,1)$”
    
    Dados reticulados complementados $(L,s,i,c,0,1)$ y $(L',s',i',c',0',1')$, diremos que $(L,s,i,c,0,1)$ es un subreticulado complementado de $(L',s',i',c',0',1')$ si se cumplen las siguientes condiciones:
    \begin{enumerate}[label=(\roman*)]
        \item $L \subseteq L'$
        \item $L$ es cerrado bajo las operaciones $s'$, $i'$ y $c'$
        \item $0 = 0'$ y $1 = 1'$
        \item $s = s'|_{L \times L}, \; i = i'|_{L \times L}, \; c = c'|_L$
    \end{enumerate}

    Sea $(L,s,i,c,0,1)$ un reticulado complementado. Un conjunto $S \subseteq L$ se llama subuniverso de $(L,s,i,c,0,1)$ si $0,1 \in S$ y además $S$ es cerrado bajo las operaciones $s$, $i$ y $c$.
\end{enumerate}

\section*{Combo 10}
\begin{enumerate}[label=(\arabic*)]
    \item Defina “tesis del bloque $\langle i, j \rangle$ en $(\varphi, J)$”
    
    \bigskip
    Sea $(\varphi, J)$ un par adecuado de tipo $\tau$ y $\langle i, j \rangle \in B^J$.  
    La \emph{tesis} del bloque $\langle i, j \rangle$ en $(\varphi, J)$ es la sentencia $\varphi_j$.

    \item Defina cuando una teoría de primer orden $(\Sigma, \tau)$ es consistente

    \bigskip
    Una teoría $(\Sigma, \tau)$ será \emph{inconsistente} cuando exista una sentencia $\varphi$ tal que
    \[
    (\Sigma, \tau) \vdash (\varphi \wedge \neg \varphi).
    \] 
    Una teoría $(\Sigma, \tau)$ será \emph{consistente} cuando no sea inconsistente.

    \item Dada una teoría elemental $(\Sigma, \tau)$ y una sentencia elemental pura $\varphi$ de tipo $\tau$, defina “prueba elemental de $\varphi$ en $(\Sigma, \tau)$”
    
    Una prueba elemental de $\varphi$ en $(\Sigma, \tau)$ es una prueba de $\varphi$ que cumple las siguientes características:
    \begin{enumerate}[label=(\arabic*)]
        \item En la prueba se parte de una estructura de tipo $\tau$, fija pero arbitraria, en el sentido de que lo único que sabemos es que satisface los axiomas de $\Sigma$ (esta es la única información particular que podemos usar).
        \item Las deducciones en la prueba son muy simples y obvias de justificar con mínimas frases en castellano.
        \item En la escritura de la prueba, lo concerniente a la matemática misma se expresa usando solo sentencias elementales de tipo $\tau$.
    \end{enumerate}
\end{enumerate}

\section*{Combo 11}
Enuncie el programa de lógica matemática dado al final de la Guía 8 y explique brevemente con qué definiciones matemáticas se van resolviendo los tres primeros puntos y qué teoremas garantizan la resolución del cuarto punto de dicho programa.

\textbf{Programa de Lógica Matemática:}
\begin{enumerate}[label=(\arabic*)]
    \item Dar un modelo matemático del concepto de fórmula elemental de tipo $\tau$.
    \item Dar una definición matemática de cuándo una fórmula elemental de tipo $\tau$ es verdadera en una estructura de tipo $\tau$ para una asignación dada de valores a las variables libres y a los nombres de elementos fijos de dicha fórmula elemental.
    \item (Plato gordo) Dar un modelo matemático del concepto de prueba elemental en una teoría elemental. A estos objetos matemáticos los llamaremos \emph{pruebas formales}.
    \item (Sublime) Intentar probar matemáticamente que nuestro concepto de prueba formal es una correcta modelización matemática de la idea intuitiva de prueba elemental en una teoría elemental.
\end{enumerate}
\medskip
\textbf{Resolución del punto (1):}  
Si $\tau = (C,F,R,a)$ es un tipo, diremos que $\tau'$ es una extensión de $\tau$ por nombres de constantes si $\tau'$ es de la forma $(C',F,R,a)$ con $C \subseteq C'$.  

Las fórmulas de tipo $\tau$ son un modelo matemático de las fórmulas elementales puras de tipo $\tau$ (es decir, aquellas en las que no ocurren nombres de elementos fijos).  
Los nombres de elementos fijos se usan en las pruebas elementales para denotar elementos fijos (a veces arbitrarios y otras veces que cumplen alguna propiedad). Cuando un matemático realiza una prueba elemental en una teoría elemental $(\Sigma, \tau)$, comienza imaginando una estructura de tipo $\tau$ de la cual lo único que sabe es que cumple las sentencias de $\Sigma$. Luego, cuando fija un elemento y le pone nombre, digamos $b$, podemos pensar que expandió su estructura imaginaria a una de tipo $(C \cup \{b\}, F, R, a)$ y continúa su razonamiento. Esta mecánica de prueba del matemático puede repetirse varias veces a lo largo de la prueba, por lo cual su estructura imaginaria va cambiando de tipo.  
Esta mecánica justifica modelizar las fórmulas elementales de tipo $\tau$ con fórmulas de tipo $\tau_1$, donde $\tau_1$ es alguna extensión de $\tau$ por nombres de constantes.

\medskip
\textbf{Resolución del punto (2):}  
La definición matemática de la relación $A \vDash \varphi[\vec{a}]$ soluciona el punto (2).

\medskip
\textbf{Resolución del punto (3):}  
La definición de prueba formal en una teoría de primer orden soluciona el punto (3).

\medskip
\textbf{Resolución del punto (4):}  
Este punto es resuelto por los teoremas de Corrección y Completitud.  
El Teorema de Corrección asegura que nuestro concepto de prueba formal no es demasiado permisivo como para permitir probar sentencias que son falsas en algún modelo de la teoría.  
Sin embargo, dicho concepto podría ser incompleto en el sentido de que podría pasar que un matemático diera una prueba elemental de una sentencia $\varphi$ en una teoría $(\Sigma, \tau)$ pero que no haya una prueba formal de $\varphi$ en $(\Sigma, \tau)$.  
El Teorema de Completitud de Gödel garantiza que este no es el caso.

\section*{Combo 12}
Defina el concepto de función y desarrolle las tres Convenciones Notacionales asociadas a dicho concepto (Guía 0).

\bigskip

\textbf{Definición de función.}  
Una \emph{función} es un conjunto de pares ordenados 
\[
f \subseteq A \times B
\]
que cumple la siguiente propiedad (F):
\[
(F) \quad \text{Si } (x, y) \in f \text{ y } (x, z) \in f, \text{ entonces } y = z.
\]
Esto significa que cada elemento del dominio \(x\) se asocia con un único valor \(y\).  
Así, una función puede verse como una \emph{regla de correspondencia} que asigna a cada elemento \(x\) de un conjunto (el \textbf{dominio}) un único elemento \(y\) de otro conjunto (la \textbf{imagen}, contenida en un conjunto de llegada).

\bigskip

\textbf{Convención Notacional 1.}  
Dado \(x \in D_f\), se escribe \(f(x)\) para denotar el único \(y \in I_f\) tal que \((x, y) \in f\).

\[
f(x) = y \quad \text{si y sólo si} \quad (x, y) \in f.
\]

\bigskip

\textbf{Convención Notacional 2.}  
Se escribe
\[
f : S \subseteq A \to B
\]
para indicar que \(f\) es una función con \(D_f = S \subseteq A\) e \(I_f \subseteq B\).  
También puede escribirse \(f : A \to B\) cuando \(D_f = A\).  

En este contexto:
\begin{itemize}
  \item \(A\): conjunto de partida.
  \item \(B\): conjunto de llegada (no necesariamente igual a la imagen, sólo debe contenerla: \(I_f \subseteq B\)).
\end{itemize}

\bigskip

\textbf{Convención Notacional 3.}  
Muchas veces para definir una función \emph{f}, lo haremos dando su dominio y su
regla de asignacion, es decir especificaremos en forma precisa que conjunto
es el dominio de f y ademas especificaremos en forma precisa quien es f(x)
para cada x de dicho dominio. Obviamente esto determina por completo
a la funcion f ya que siempre se da que ya que:
\[
f = \{(x, f(x)) : x \in D_f\}.
\]

Usualmente se escribe de forma más intuitiva:
\[
f : D_f \to B, \quad x \mapsto f(x).
\]

\end{document}
