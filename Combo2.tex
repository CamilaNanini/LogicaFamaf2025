\documentclass[a4paper,12pt]{article}
\usepackage[spanish]{babel}
\usepackage[utf8]{inputenc}
\usepackage[T1]{fontenc}
\usepackage{amsmath, amssymb, amsthm}
\usepackage{enumitem}

\title{Combos de Definiciones}
\author{Camila Nanini}

\newtheorem{theorem}{Teorema}
\newtheorem{lemma}{Lema}
\newcommand{\deduce}[2]{(#1, #2) \vdash}
\newcommand{\incon}{es inconsistente}

\begin{document}

\maketitle

\section*{Combo 2}

\begin{theorem}[de Dedekind]
Sea $(L, s, i)$ un reticulado terna. La relación binaria definida por:
\[
x \le y \quad \text{si y sólo si} \quad x s y = y
\]
es un orden parcial sobre $L$ para el cual se cumple que:
\[
\sup(\{x, y\}) = x s y, \quad \inf(\{x, y\}) = x i y
\]
cualesquiera sean $x, y \in L$.

\begin{proof}
Primero probaremos la reflexividad de \(\le\). Para todo \(x\in L\) debemos mostrar \(x\le x\), es decir
\[
x \;s\; x = x.
\]
Ésta es precisamente la identidad de idempotencia para \(s\) (una de las igualdades axiomáticas del reticulado), por lo que \(x\le x\) para todo \(x\in L\). Entonces \(\le\) es reflexiva.

Ahora probaremos la antisimetría. Sean \(x,y\in L\) tales que \(x\le y\) y \(y\le x\). Por la definición de \(\le\) esto equivale a
\[
x\; s\; y = y \qquad\text{y}\qquad y\; s\; x = x.
\]
Pero \(s\) es conmutativa, luego \(x\; s\; y = y\; s\; x\). Combinando las igualdades anteriores obtenemos
\[
y = x\; s\; y = y\; s\; x = x,
\]
es decir \(x=y\). Por tanto \(\le\) es antisimétrica.
\end{proof}

\begin{proof}
Veamos que \(\le\) es transitiva con respecto a \(L\). Supongamos que \(x \le y\) e \(y \le z\). Es decir, que por definición de \(\le\) tenemos que
\[
x \; s \; y = y \qquad \text{y} \qquad y \; s \; z = z.
\]
Entonces
\[
x \; s \; z 
= x \; s \; (y \; s \; z)
= (x \; s \; y) \; s \; z
= y \; s \; z
= z,
\]
por lo cual \(x \le z\).  
O sea que ya sabemos que \((L, \le)\) es un poset.

Veamos ahora que \(\sup(\{x, y\}) = x \; s \; y\).  
Primero debemos ver que \(x \; s \; y\) es una cota superior del conjunto \(\{x, y\}\), es decir:
\[
x \le x \; s \; y \qquad \text{y} \qquad y \le x \; s \; y.
\]
Por la definición de \(\le\), debemos probar que
\[
x \; s \; (x \; s \; y) = x \; s \; y 
\qquad \text{y} \qquad 
y \; s \; (x \; s \; y) = x \; s \; y.
\]
Estas igualdades se pueden probar usando las propiedades (I1), (I2) y (I4).

Nos falta ver entonces que \(x \; s \; y\) es menor o igual que cualquier cota superior de \(\{x, y\}\).  
Supongamos \(x, y \le z\). Es decir que, por definición de \(\le\), tenemos que
\[
x \; s \; z = z 
\qquad \text{y} \qquad 
y \; s \; z = z.
\]
Pero entonces
\[
(x \; s \; y) \; s \; z 
= x \; s \; (y \; s \; z)
= x \; s \; z
= z,
\]
por lo que \(x \; s \; y \le z\).  
Es decir que \(x \; s \; y\) es la menor cota superior.

Para probar que \(\inf(\{x, y\}) = x \; i \; y\), probaremos que para todo \(u, v \in L\),
\[
u \le v \quad \text{si y sólo si} \quad u \; i \; v = u,
\]
lo cual le permitirá al lector aplicar un razonamiento similar al usado en la prueba de que \(\sup(\{x, y\}) = x \; s \; y\).

Supongamos que \(u \le v\). Por definición tenemos que \(u \; s \; v = v\). Entonces
\[
u \; i \; v = u \; i \; (u \; s \; v).
\]
Pero por (I7) tenemos que \(u \; i \; (u \; s \; v) = u\), lo cual implica \(u \; i \; v = u\).

Recíprocamente, si \(u \; i \; v = u\), entonces
\[
u \; s \; v 
= (u \; i \; v) \; s \; v
= v \; s \; (u \; i \; v) \quad \text{(por (I2))} 
= v \; s \; (v \; i \; u) \quad \text{(por (I3))} 
= v \quad \text{(por (I6))}.
\]
Lo cual nos dice que \(u \le v\).
\end{proof}
\end{theorem}

\begin{lemma}
Supongamos que $\vec{a}, \vec{b}$ son asignaciones tales que si $x_i \in Li(\varphi)$, entonces $a_i = b_i$. Entonces:
\[
A \vDash \varphi[\vec{a}] \text{ si y sólo si } A \vDash \varphi[\vec{b}].
\]

\begin{proof}
Sea \(A=(A,i)\) una estructura de tipo \(\tau\). Queremos probar el caso base, hay dos subcasos:

\medskip
\noindent\textbf{Caso 1: } \(\varphi\) es una igualdad atómica \(t\equiv s\).

Supongamos que \(\vec a,\vec b\) coinciden en todas las variables que ocurren en \(\varphi\).
En particular, todas las variables que ocurren en \(t\) y en \(s\) son variables de
\(Li(\varphi)\). Así que tenemos
\[
t^A[\vec a]=t^A[\vec b]\quad\text{y}\quad s^A[\vec a]=s^A[\vec b].
\]
Esto por lema auxiliar 1.
Entonces
\[
\begin{aligned}
A\models (t\equiv s)[\vec a]
&\iff t^A[\vec a]=s^A[\vec a]\qquad &&\text{(definición de satisfacción para igualdad)}\\
&\iff t^A[\vec b]=s^A[\vec b]\qquad &&\text{(por las igualdades anteriores)}\\
&\iff A\models (t\equiv s)[\vec b].
\end{aligned}
\]

\medskip
\noindent\textbf{Caso 2: } \(\varphi\) es una fórmula atómica de predicado
\(r(t_1,\dots,t_m)\).

Nuevamente por lema auxiliar 1 se da que en cada \(t_j\):
\[
t_j^A[\vec a]=t_j^A[\vec b]\qquad\text{para }j=1,\dots,m.
\]
Entonces las \(m\)-uplas de valores de los términos coinciden:
\[
\big(t_1^A[\vec a],\dots,t_m^A[\vec a]\big)=\big(t_1^A[\vec b],\dots,t_m^A[\vec b]\big).
\]
Por la definición \(\models\)
\[
\begin{aligned}
A\models r(t_1,\dots,t_m)[\vec a]
&\iff \big(t_1^A[\vec a],\dots,t_m^A[\vec a]\big)\in i(r)\\
&\iff \big(t_1^A[\vec b],\dots,t_m^A[\vec b]\big)\in i(r)\\
&\iff A\models r(t_1,\dots,t_m)[\vec b].
\end{aligned}
\]

\medskip
\noindent
Veamos que Teo$_k$ implica Teo$_{k+1}$. Sea 
$\varphi \in F_\tau^{k+1} - F_\tau^k$. 
Hay varios casos:

\medskip
\noindent \textbf{Caso} $\varphi = (\varphi_1 \wedge \varphi_2)$.

\noindent
Ya que $L_i(\varphi_i) \subseteq L_i(\varphi)$, $i = 1,2$, Teo$_k$ nos dice que 
$A \vDash \varphi_i[\vec{a}]$ sii $A \vDash \varphi_i[\vec{b}]$, para $i = 1,2$. 
Se tiene entonces que:
\[
A \vDash \varphi[\vec{a}]
\iff \text{(por (3) en la def. de $A \vDash \varphi[\vec{a}]$)}
A \vDash \varphi_1[\vec{a}] \text{ y } A \vDash \varphi_2[\vec{a}]
\]
\[
\iff \text{(por Teo$_k$)}
A \vDash \varphi_1[\vec{b}] \text{ y } A \vDash \varphi_2[\vec{b}]
\]
\[
\iff \text{(por (3) en la def. de $A \vDash \varphi[\vec{a}]$)}
A \vDash \varphi[\vec{b}]
\]

\medskip
\noindent \textbf{Caso} $\varphi = (\varphi_1 \vee \varphi_2)$.

\noindent
Es completamente similar al anterior.

\medskip
\noindent \textbf{Caso} $\varphi = (\varphi_1 \to \varphi_2)$.

\noindent
Es completamente similar al anterior.

\medskip
\noindent \textbf{Caso} $\varphi = (\varphi_1 \leftrightarrow \varphi_2)$.

\noindent
Es completamente similar al anterior.

\medskip
\noindent \textbf{Caso} $\varphi = \neg \varphi_1$.

\noindent
Es completamente similar al anterior.

\medskip
\noindent \textbf{Caso} $\varphi = \forall x_j \varphi_1$.

\noindent
Supongamos $A \vDash \varphi[\vec{a}]$. Entonces, por (8) en la definición de 
$A \vDash \varphi[\vec{a}]$, se tiene que 
$A \vDash \varphi_1[\downarrow^a_j(\vec{a})]$, para todo $a \in A$. 
Nótese que $\downarrow^a_j(\vec{a})$ y $\downarrow^a_j(\vec{b})$ 
coinciden en toda $x_i \in L_i(\varphi_1)$, ya que 
$L_i(\varphi_1) \subseteq L_i(\varphi) \cup \{x_j\}$. 
O sea, por Teo$_k$ se tiene que 
$A \vDash \varphi_1[\downarrow^a_j(\vec{b})]$ para todo $a \in A$, 
lo cual, por (8) en la definición de $A \vDash \varphi[\vec{a}]$, 
nos dice que $A \vDash \varphi[\vec{b}]$. 
La prueba de que $A \vDash \varphi[\vec{b}]$ implica que 
$A \vDash \varphi[\vec{a}]$ es similar.

\medskip
\noindent \textbf{Caso} $\varphi = \exists x_j \varphi_1$.

\noindent
Es similar al anterior.
\end{proof}
\end{lemma}

\section*{Auxiliares de Demostraciones}
\begin{enumerate}
  
    \item Lema auxiliar 1: Sea $A$ una estructura de tipo $\tau$ y sea $t \in T_\tau$. 
    Supongamos que $\vec{a}, \vec{b}$ son asignaciones tales que 
    $a_i = b_i$ cada vez que $x_i$ ocurra en $t$. 
    Entonces 
    \[
    t^A[\vec{a}] = t^A[\vec{b}].
    \]

    \item \textbf{Identidades de los reticulados terna.}

    \begin{align*}
    \text{(I1)} \quad & x \, s \, x = x \, i \, x = x, && \text{cualesquiera sea } x \in L, \\
    \text{(I2)} \quad & x \, s \, y = y \, s \, x, && \text{cualesquiera sean } x, y \in L, \\
    \text{(I3)} \quad & x \, i \, y = y \, i \, x, && \text{cualesquiera sean } x, y \in L, \\
    \text{(I4)} \quad & (x \, s \, y) \, s \, z = x \, s \, (y \, s \, z), && \text{cualesquiera sean } x, y, z \in L, \\
    \text{(I5)} \quad & (x \, i \, y) \, i \, z = x \, i \, (y \, i \, z), && \text{cualesquiera sean } x, y, z \in L, \\
    \text{(I6)} \quad & x \, s \, (x \, i \, y) = x, && \text{cualesquiera sean } x, y \in L, \\
    \text{(I7)} \quad & x \, i \, (x \, s \, y) = x, && \text{cualesquiera sean } x, y \in L.
    \end{align*}

\end{enumerate}

\end{document}

