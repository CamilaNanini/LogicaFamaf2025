\documentclass[a4paper,12pt]{article}
\usepackage[spanish]{babel}
\usepackage[utf8]{inputenc}
\usepackage[T1]{fontenc}
\usepackage{amsmath, amssymb, amsthm}
\usepackage{enumitem}

\title{Combos de Definiciones}
\author{Camila Nanini}

\newtheorem{theorem}{Teorema}
\newtheorem{lemma}{Lema}

\begin{document}

\maketitle

\section*{Combo 1}

\begin{theorem}[del Filtro Primo]
Sea $(L, s, i)$ un reticulado terna distributivo y $F$ un filtro. Supongamos $x_0 \in L - F$. Entonces hay un filtro primo $P$ tal que
\[
x_0 \notin P \quad \text{y} \quad F \subseteq P.
\]

\end{theorem}

\begin{lemma}[Propiedades básicas de la consistencia]
Sea $(\Sigma, \tau)$ una teoría.
\begin{enumerate}[label=(\arabic*)]
    \item Si $(\Sigma, \tau)$ es inconsistente, entonces $(\Sigma, \tau) \vdash \varphi$, para toda sentencia $\varphi$.
    \item Si $(\Sigma, \tau)$ es consistente y $(\Sigma, \tau) \vdash \varphi$, entonces $(\Sigma \cup \{\varphi\}, \tau)$ es consistente.
    \item Si $(\Sigma, \tau) \not\vdash \neg\varphi$, entonces $(\Sigma \cup \{\varphi\}, \tau)$ es consistente.
\end{enumerate}
\end{lemma}

\section*{Combo 2}

\begin{theorem}[de Dedekind]
Sea $(L, s, i)$ un reticulado terna. La relación binaria definida por:
\[
x \le y \quad \text{si y sólo si} \quad x s y = y
\]
es un orden parcial sobre $L$ para el cual se cumple que:
\[
\sup(\{x, y\}) = x s y, \quad \inf(\{x, y\}) = x i y
\]
cualesquiera sean $x, y \in L$.

\begin{proof}
Primero probaremos la reflexividad de \(\le\). Para todo \(x\in L\) debemos mostrar \(x\le x\), es decir
\[
x \;s\; x = x.
\]
Ésta es precisamente la identidad de idempotencia para \(s\) (una de las igualdades axiomáticas del reticulado), por lo que \(x\le x\) para todo \(x\in L\). Entonces \(\le\) es reflexiva.

Ahora probaremos la antisimetría. Sean \(x,y\in L\) tales que \(x\le y\) y \(y\le x\). Por la definición de \(\le\) esto equivale a
\[
x\; s\; y = y \qquad\text{y}\qquad y\; s\; x = x.
\]
Pero \(s\) es conmutativa, luego \(x\; s\; y = y\; s\; x\). Combinando las igualdades anteriores obtenemos
\[
y = x\; s\; y = y\; s\; x = x,
\]
es decir \(x=y\). Por tanto \(\le\) es antisimétrica.
\end{proof}

\begin{proof}
Veamos que \(\le\) es transitiva con respecto a \(L\). Supongamos que \(x \le y\) e \(y \le z\). Es decir, que por definición de \(\le\) tenemos que
\[
x \; s \; y = y \qquad \text{y} \qquad y \; s \; z = z.
\]
Entonces
\[
x \; s \; z 
= x \; s \; (y \; s \; z)
= (x \; s \; y) \; s \; z
= y \; s \; z
= z,
\]
por lo cual \(x \le z\).  
O sea que ya sabemos que \((L, \le)\) es un poset.

Veamos ahora que \(\sup(\{x, y\}) = x \; s \; y\).  
Primero debemos ver que \(x \; s \; y\) es una cota superior del conjunto \(\{x, y\}\), es decir:
\[
x \le x \; s \; y \qquad \text{y} \qquad y \le x \; s \; y.
\]
Por la definición de \(\le\), debemos probar que
\[
x \; s \; (x \; s \; y) = x \; s \; y 
\qquad \text{y} \qquad 
y \; s \; (x \; s \; y) = x \; s \; y.
\]
Estas igualdades se pueden probar usando las propiedades (I1), (I2) y (I4).

Nos falta ver entonces que \(x \; s \; y\) es menor o igual que cualquier cota superior de \(\{x, y\}\).  
Supongamos \(x, y \le z\). Es decir que, por definición de \(\le\), tenemos que
\[
x \; s \; z = z 
\qquad \text{y} \qquad 
y \; s \; z = z.
\]
Pero entonces
\[
(x \; s \; y) \; s \; z 
= x \; s \; (y \; s \; z)
= x \; s \; z
= z,
\]
por lo que \(x \; s \; y \le z\).  
Es decir que \(x \; s \; y\) es la menor cota superior.

Para probar que \(\inf(\{x, y\}) = x \; i \; y\), probaremos que para todo \(u, v \in L\),
\[
u \le v \quad \text{si y sólo si} \quad u \; i \; v = u,
\]
lo cual le permitirá al lector aplicar un razonamiento similar al usado en la prueba de que \(\sup(\{x, y\}) = x \; s \; y\).

Supongamos que \(u \le v\). Por definición tenemos que \(u \; s \; v = v\). Entonces
\[
u \; i \; v = u \; i \; (u \; s \; v).
\]
Pero por (I7) tenemos que \(u \; i \; (u \; s \; v) = u\), lo cual implica \(u \; i \; v = u\).

Recíprocamente, si \(u \; i \; v = u\), entonces
\[
u \; s \; v 
= (u \; i \; v) \; s \; v
= v \; s \; (u \; i \; v) \quad \text{(por (I2))} 
= v \; s \; (v \; i \; u) \quad \text{(por (I3))} 
= v \quad \text{(por (I6))}.
\]
Lo cual nos dice que \(u \le v\).
\end{proof}
\end{theorem}

\begin{lemma}
Supongamos que $\vec{a}, \vec{b}$ son asignaciones tales que si $x_i \in Li(\varphi)$, entonces $a_i = b_i$. Entonces:
\[
A \vDash \varphi[\vec{a}] \text{ si y sólo si } A \vDash \varphi[\vec{b}].
\]

\begin{proof}
Sea \(A=(A,i)\) una estructura de tipo \(\tau\). Queremos probar el caso base, hay dos subcasos:

\medskip
\noindent\textbf{Caso 1: } \(\varphi\) es una igualdad atómica \(t\equiv s\).

Supongamos que \(\vec a,\vec b\) coinciden en todas las variables que ocurren en \(\varphi\).
En particular, todas las variables que ocurren en \(t\) y en \(s\) son variables de
\(Li(\varphi)\). Así que tenemos
\[
t^A[\vec a]=t^A[\vec b]\quad\text{y}\quad s^A[\vec a]=s^A[\vec b].
\]
Esto por lema auxiliar.
Entonces
\[
\begin{aligned}
A\models (t\equiv s)[\vec a]
&\iff t^A[\vec a]=s^A[\vec a]\qquad &&\text{(definición de satisfacción para igualdad)}\\
&\iff t^A[\vec b]=s^A[\vec b]\qquad &&\text{(por las igualdades anteriores)}\\
&\iff A\models (t\equiv s)[\vec b].
\end{aligned}
\]

\medskip
\noindent\textbf{Caso 2: } \(\varphi\) es una fórmula atómica de predicado
\(r(t_1,\dots,t_m)\).

Nuevamente por lema auxiliar se da que en cada \(t_j\):
\[
t_j^A[\vec a]=t_j^A[\vec b]\qquad\text{para }j=1,\dots,m.
\]
Entonces las \(m\)-uplas de valores de los términos coinciden:
\[
\big(t_1^A[\vec a],\dots,t_m^A[\vec a]\big)=\big(t_1^A[\vec b],\dots,t_m^A[\vec b]\big).
\]
Por la definición \(\models\)
\[
\begin{aligned}
A\models r(t_1,\dots,t_m)[\vec a]
&\iff \big(t_1^A[\vec a],\dots,t_m^A[\vec a]\big)\in i(r)\\
&\iff \big(t_1^A[\vec b],\dots,t_m^A[\vec b]\big)\in i(r)\\
&\iff A\models r(t_1,\dots,t_m)[\vec b].
\end{aligned}
\]

\medskip
\noindent
Veamos que Teo$_k$ implica Teo$_{k+1}$. Sea 
$\varphi \in F_\tau^{k+1} - F_\tau^k$. 
Hay varios casos:

\medskip
\noindent \textbf{Caso} $\varphi = (\varphi_1 \wedge \varphi_2)$.

\noindent
Ya que $L_i(\varphi_i) \subseteq L_i(\varphi)$, $i = 1,2$, Teo$_k$ nos dice que 
$A \vDash \varphi_i[\vec{a}]$ sii $A \vDash \varphi_i[\vec{b}]$, para $i = 1,2$. 
Se tiene entonces que:
\[
A \vDash \varphi[\vec{a}]
\iff \text{(por (3) en la def. de $A \vDash \varphi[\vec{a}]$)}
A \vDash \varphi_1[\vec{a}] \text{ y } A \vDash \varphi_2[\vec{a}]
\]
\[
\iff \text{(por Teo$_k$)}
A \vDash \varphi_1[\vec{b}] \text{ y } A \vDash \varphi_2[\vec{b}]
\]
\[
\iff \text{(por (3) en la def. de $A \vDash \varphi[\vec{a}]$)}
A \vDash \varphi[\vec{b}]
\]

\medskip
\noindent \textbf{Caso} $\varphi = (\varphi_1 \vee \varphi_2)$.

\noindent
Es completamente similar al anterior.

\medskip
\noindent \textbf{Caso} $\varphi = (\varphi_1 \to \varphi_2)$.

\noindent
Es completamente similar al anterior.

\medskip
\noindent \textbf{Caso} $\varphi = (\varphi_1 \leftrightarrow \varphi_2)$.

\noindent
Es completamente similar al anterior.

\medskip
\noindent \textbf{Caso} $\varphi = \neg \varphi_1$.

\noindent
Es completamente similar al anterior.

\medskip
\noindent \textbf{Caso} $\varphi = \forall x_j \varphi_1$.

\noindent
Supongamos $A \vDash \varphi[\vec{a}]$. Entonces, por (8) en la definición de 
$A \vDash \varphi[\vec{a}]$, se tiene que 
$A \vDash \varphi_1[\downarrow a_j(\vec{a})]$, para todo $a \in A$. 
Nótese que $\downarrow a_j(\vec{a})$ y $\downarrow a_j(\vec{b})$ 
coinciden en toda $x_i \in L_i(\varphi_1)$, ya que 
$L_i(\varphi_1) \subseteq L_i(\varphi) \cup \{x_j\}$. 
O sea, por Teo$_k$ se tiene que 
$A \vDash \varphi_1[\downarrow a_j(\vec{b})]$ para todo $a \in A$, 
lo cual, por (8) en la definición de $A \vDash \varphi[\vec{a}]$, 
nos dice que $A \vDash \varphi[\vec{b}]$. 
La prueba de que $A \vDash \varphi[\vec{b}]$ implica que 
$A \vDash \varphi[\vec{a}]$ es similar.

\medskip
\noindent \textbf{Caso} $\varphi = \exists x_j \varphi_1$.

\noindent
Es similar al anterior.
\end{proof}

Teorema auxiliar: Sea $A$ una estructura de tipo $\tau$ y sea $t \in T_\tau$. 
Supongamos que $\vec{a}, \vec{b}$ son asignaciones tales que 
$a_i = b_i$ cada vez que $x_i$ ocurra en $t$. 
Entonces 
\[
t^A[\vec{a}] = t^A[\vec{b}].
\]
\begin{proof}
\noindent
Caso base Teo$_0$, tiene dos subcasos. 
\medskip

\noindent\textbf{Caso 1: } \(t\) es una variable, digamos \(t = x_k\).

Por definición del valor de un término en la estructura \(A\) para una asignación \(\vec a\),
\[
t^A[\vec a] = a_k,
\]
y de forma análoga
\[
t^A[\vec b] = b_k.
\]
Por la hipótesis del enunciado, si \(x_k\) ocurre en \(t\) (y aquí \(x_k\) es precisamente el término), entonces \(a_k=b_k\).
Por tanto \(t^A[\vec a]=a_k=b_k=t^A[\vec b]\), como se quería demostrar.

\medskip
\noindent\textbf{Caso 2: } \(t\) es una constante, digamos \(t=c\in C\).

Por definición del valor de una constante en la estructura \(A=(A,i)\),
\[
t^A[\vec a] = c^A = i(c),
\]
y análogamente
\[
t^A[\vec b] = c^A = i(c).
\]
Así \(t^A[\vec a]=i(c)=t^A[\vec b]\).

\medskip
\noindent
Veamos que Teo$_k \Rightarrow$ Teo$_{k+1}$. Supongamos $t \in T_\tau^{k+1} - T_\tau^k$ y sean $\vec{a}, \vec{b}$ asignaciones tales que 
$a_i = b_i$ cada vez que $x_i$ ocurra en $t$.

\noindent
Nótese que $t = f(t_1, \ldots, t_n)$, con $f \in F_n$, $n \ge 1$, y $t_1, \ldots, t_n \in T_\tau^k$.

\noindent
Para cada $j = 1, \ldots, n$, tenemos que $a_i = b_i$ cada vez que $x_i$ ocurra en $t_j$, 
lo cual, por hipótesis inductiva (Teo$_k$), nos dice que
\[
t_j^A[\vec{a}] = t_j^A[\vec{b}], \quad j = 1, \ldots, n.
\]

\noindent
Se tiene entonces que
\[
\begin{aligned}
t^A[\vec{a}] 
&= i(f)\big(t_1^A[\vec{a}], \ldots, t_n^A[\vec{a}]\big)
&& \text{(por definición de $t^A[\vec{a}]$)} \\[4pt]
&= i(f)\big(t_1^A[\vec{b}], \ldots, t_n^A[\vec{b}]\big)
&& \text{(por hipótesis inductiva)} \\[4pt]
&= t^A[\vec{b}]
&& \text{(por definición de $t^A[\vec{b}]$).}
\end{aligned}
\]
\end{proof}
\end{lemma}

\section*{Combo 3}

\begin{theorem}[Lectura única de términos]
Dado $t \in T^\tau$, se da una de las siguientes:
\begin{enumerate}[label=(\arabic*)]
    \item $t \in Var \cup C$
    \item Hay únicos $n \ge 1$, $f \in F^n$, $t_1, \ldots, t_n \in T^\tau$ tales que $t = f(t_1, \ldots, t_n)$.
\end{enumerate}
\begin{proof}
\noindent
Por la definición de $T^\tau$ está claro que vale sin la unicidad (1)
En virtud del Lema de Menú de términos solo nos falta probar la unicidad en el punto (2). 
Supongamos
\[
t = f(t_1,\ldots,t_n) = g(s_1,\ldots,s_m)
\]
con $n,m\ge 1$, $f\in F_n$, $g\in F_m$, $t_1,\ldots,t_n,s_1,\ldots,s_m\in T_\tau$.
Nótese que $f=g$. Es decir, $n=m=a(f)$. 

\noindent
Nótese que $t_1$ es tramo inicial de $s_1$ o $s_1$ es tramo inicial de $t_1$, lo cual, por el lema de mordisqueo de términos, nos dice que $t_1=s_1$. Con el mismo razonamiento se prueba que necesariamente
\[
t_2=s_2,\ \ldots,\ t_n=s_n.
\]
\end{proof}
\end{theorem}

\begin{lemma}
Supongamos que $F: A \to B$ es un isomorfismo. Sea $\varphi \in F^\tau$. Entonces
\[
A \vDash \varphi[(a_1, a_2, \ldots)] \text{ si y sólo si } B \vDash \varphi[(F(a_1), F(a_2), \ldots)]
\]
para cada $(a_1, a_2, \ldots) \in A^N$. En particular, $A$ y $B$ satisfacen las mismas sentencias de tipo $\tau$.
\begin{proof}
Para $\vec{a} = (a_1, a_2, \ldots) \in A^N$, denotemos $(F(a_1), F(a_2), \ldots)$ con $F(\vec{a})$.  
Procedemos por inducción.

\medskip
\noindent
\textbf{Teo$_k$:} Supongamos que $F: A \to B$ es un isomorfismo. Sea $\varphi \in F_\tau^k$. Entonces
\[
A \vDash \varphi[\vec{a}] \text{ sii } B \vDash \varphi[F(\vec{a})],
\]
para cada $(a_1,a_2,\ldots) \in A^N$.

\medskip
\noindent
\textbf{Prueba de Teo$_0$.}  
Hay dos casos.  
\smallskip

\noindent
\textbf{Caso} $\varphi = r(t_1, \ldots, t_n)$, con $n \ge 1$, $r \in R_n$ y $t_1, \ldots, t_n \in T^\tau$.  
Tenemos entonces:
\[
\begin{aligned}
A \vDash \varphi[\vec{a}]
&\text{ sii } (t_1^A[\vec{a}], \ldots, t_n^A[\vec{a}]) \in r^A && \text{(def. de $\vDash$)}\\
&\text{ sii } (F(t_1^A[\vec{a}]), \ldots, F(t_n^A[\vec{a}])) \in r^B && \text{($F$ es iso)}\\
&\text{ sii } (t_1^B[F(\vec{a})], \ldots, t_n^B[F(\vec{a})]) \in r^B && \text{(por lema previo)}\\
&\text{ sii } B \vDash \varphi[F(\vec{a})].
\end{aligned}
\]

\textbf{Caso} $\varphi = (t \equiv s)$ con $t,s \in T^\tau$. Tenemos ahora que 
\[
\begin{aligned}
A \vDash \varphi[\vec{a}]
&\text{ sii } (t^A[\vec{a}] \equiv s^A[\vec{a}]) && \text{(def. de $\vDash$)}\\
&\text{ sii } (F(t^A[\vec{a}]) \equiv F(s^A[\vec{a}])) && \text{($F$ es iso)}\\
&\text{ sii } (t^B[F(\vec{a})] \equiv s^B[F(\vec{a})]) && \text{(por lema previo)}\\
&\text{ sii } B \vDash \varphi[F(\vec{a})].
\end{aligned}
\]
\medskip
\noindent
Veamos ahora que Teo$_k$ implica Teo$_{k+1}$.  
Supongamos que vale Teo$_k$. Probaremos que entonces vale Teo$_{k+1}$.

\smallskip
Si $\varphi \in F_\tau^k$, podemos aplicar directamente Teo$_k$.  
Supongamos entonces que $\varphi \in F_\tau^{k+1} - F_\tau^k$.  
Por el Lema de Lectura Única de fórmulas, hay varios casos.

\medskip
\noindent
\textbf{Caso} $\varphi = (\varphi_1 \lor \varphi_2)$, con $\varphi_1, \varphi_2 \in F_\tau^k$. Entonces:
\[
\begin{aligned}
A \vDash \varphi[\vec{a}]
&\text{ sii } A \vDash \varphi_1[\vec{a}] \text{ o } A \vDash \varphi_2[\vec{a}] && \text{(def. de $\vDash$)}\\
&\text{ sii } B \vDash \varphi_1[F(\vec{a})] \text{ o } B \vDash \varphi_2[F(\vec{a})] && \text{(Teo$_k$)}\\
&\text{ sii } B \vDash \varphi[F(\vec{a})] && \text{(def. de $\vDash$)}.
\end{aligned}
\]

Los casos $\varphi = (\varphi_1 \land \varphi_2)$, $\varphi = (\varphi_1 \to \varphi_2)$, 
$\varphi = (\varphi_1 \leftrightarrow \varphi_2)$ y $\varphi = \neg \varphi_1$ son análogos al anterior.

\medskip
\noindent
\textbf{Caso} $\varphi = \forall x_j \varphi_1$, con $\varphi_1 \in F_\tau^k$.  
Veamos cada implicación por separado.

\smallskip
\noindent
Supongamos $A \vDash \varphi[\vec{a}]$.  
Entonces, por la definición de $\vDash$, se tiene que 
\[
A \vDash \varphi_1[\downarrow a_j(\vec{a})], \quad \text{para todo } a \in A.
\]
Por Teo$_k$ tenemos que 
\[
B \vDash \varphi_1[F(\downarrow a_j(\vec{a}))], \quad \text{para todo } a \in A.
\]
Pero como
\[
F(\downarrow a_j(\vec{a})) = \downarrow F(a)_j(F(\vec{a})),
\]
tenemos que 
\[
B \vDash \varphi_1[\downarrow F(a)_j(F(\vec{a}))], \quad \text{para todo } a \in A.
\]
Como $F$ es sobreyectiva, obtenemos que 
\[
B \vDash \varphi_1[\downarrow b_j(F(\vec{a}))], \quad \text{para todo } b \in B.
\]
Ahora, por la definición de $\vDash$, tenemos que 
\[
B \vDash \forall x_j \varphi_1[F(\vec{a})],
\]
es decir, $B \vDash \varphi[F(\vec{a})]$.

\smallskip
\noindent
Recíprocamente, supongamos que $B \vDash \varphi[F(\vec{a})]$.  
La definición de $\vDash$ nos dice que 
\[
B \vDash \varphi_1[\downarrow b_j(F(\vec{a}))], \quad \text{para todo } b \in B.
\]
Obviamente, esto implica que 
\[
B \vDash \varphi_1[\downarrow F(a)_j(F(\vec{a}))], \quad \text{para todo } a \in A.
\]
Pero como 
\[
\downarrow F(a)_j(F(\vec{a})) = F(\downarrow a_j(\vec{a})),
\]
tenemos que 
\[
B \vDash \varphi_1[F(\downarrow a_j(\vec{a}))], \quad \text{para todo } a \in A.
\]
Por Teo$_k$, se sigue que 
\[
A \vDash \varphi_1[\downarrow a_j(\vec{a})], \quad \text{para todo } a \in A,
\]
lo cual, por la definición de $\vDash$, nos dice que $A \vDash \varphi[\vec{a}]$.

\medskip
\noindent
El caso $\varphi = \exists x_j \varphi_1$ es análogo al anterior.
\end{proof}
\end{lemma}

\begin{theorem}
Sea $T = (\Sigma, \tau)$ una teoría. Entonces
\[
(S^\tau / \dashv\vdash_T, s^T, i^T, c^T, 0^T, 1^T)
\]
es un álgebra de Boole. \\

\textbf{Pruebe sólo el ítem (6).}
\begin{proof}
Veamos que
\[
[\varphi_1]_T \, s_T \, \big( [\varphi_2]_T \, s_T \, [\varphi_3]_T \big)
= \big( [\varphi_1]_T \, s_T \, [\varphi_2]_T \big) \, s_T \, [\varphi_3]_T,
\]
cualesquiera sean $\varphi_1, \varphi_2, \varphi_3 \in S_\tau$.

Sean $\varphi_1, \varphi_2, \varphi_3 \in S_\tau$ fijas.  
Por la definición de la operación $s_T$ tenemos que:
\[
\begin{aligned}
[\varphi_1]_T \, s_T \, \big( [\varphi_2]_T \, s_T \, [\varphi_3]_T \big)
&= [\varphi_1]_T \, s_T \, [(\varphi_2 \vee \varphi_3)]_T \\
&= [(\varphi_1 \vee (\varphi_2 \vee \varphi_3))]_T, \\[6pt]
\big( [\varphi_1]_T \, s_T \, [\varphi_2]_T \big) \, s_T \, [\varphi_3]_T
&= [(\varphi_1 \vee \varphi_2)]_T \, s_T \, [\varphi_3]_T \\
&= [((\varphi_1 \vee \varphi_2) \vee \varphi_3)]_T.
\end{aligned}
\]

\noindent
Por tanto, debemos probar que
\[
[(\varphi_1 \vee (\varphi_2 \vee \varphi_3))]_T
= [((\varphi_1 \vee \varphi_2) \vee \varphi_3)]_T,
\]
es decir, que
\[
T \vdash 
\big( (\varphi_1 \vee (\varphi_2 \vee \varphi_3)) 
\leftrightarrow 
((\varphi_1 \vee \varphi_2) \vee \varphi_3) \big).
\]

\noindent
Nótese que, por (2) del Lema 7.38, basta con probar que:
\[
T \vdash 
\big( (\varphi_1 \vee (\varphi_2 \vee \varphi_3))
\rightarrow 
((\varphi_1 \vee \varphi_2) \vee \varphi_3) \big),
\qquad
T \vdash 
\big( ((\varphi_1 \vee \varphi_2) \vee \varphi_3)
\rightarrow 
(\varphi_1 \vee (\varphi_2 \vee \varphi_3)) \big).
\]

\noindent
A continuación damos una prueba formal de
\[
(\varphi_1 \vee (\varphi_2 \vee \varphi_3))
\rightarrow 
((\varphi_1 \vee \varphi_2) \vee \varphi_3)
\text{ en } T,
\]

\begin{enumerate}[label=\arabic*.]
    \item $(\varphi_1 \vee (\varphi_2 \vee \varphi_3))$ \hfill \textit{Hipótesis 1}
    \item $\varphi_1$ \hfill \textit{Hipótesis 2}
    \item $(\varphi_1 \vee \varphi_2)$ \hfill \textit{Introducción de $\vee$ (2)}
    \item $((\varphi_1 \vee \varphi_2) \vee \varphi_3)$ \hfill \textit{Tesis 2, Introducción de $\vee$ (3)}
    \item $\varphi_1 \rightarrow ((\varphi_1 \vee \varphi_2) \vee \varphi_3)$ \hfill \textit{Conclusión}
    \item $(\varphi_2 \vee \varphi_3)$ \hfill \textit{Hipótesis 3}
    \item $\varphi_2$ \hfill \textit{Hipótesis 4}
    \item $(\varphi_1 \vee \varphi_2)$ \hfill \textit{Introducción de $\vee$ (6)}
    \item $((\varphi_1 \vee \varphi_2) \vee \varphi_3)$ \hfill \textit{Tesis 4, Introducción de $\vee$ (7)}
    \item $\varphi_2 \rightarrow ((\varphi_1 \vee \varphi_2) \vee \varphi_3)$ \hfill \textit{Conclusión}
    \item $\varphi_3$ \hfill \textit{Hipótesis 5}
    \item $((\varphi_1 \vee \varphi_2) \vee \varphi_3)$ \hfill \textit{Tesis 5, Introducción de $\vee$ (11)}
    \item $\varphi_3 \rightarrow ((\varphi_1 \vee \varphi_2) \vee \varphi_3)$ \hfill \textit{Conclusión}
    \item $((\varphi_1 \vee \varphi_2) \vee \varphi_3)$ \hfill \textit{Tesis 3, División por casos (6, 10, 13)}
    \item $(\varphi_2 \vee \varphi_3) \rightarrow ((\varphi_1 \vee \varphi_2) \vee \varphi_3)$ \hfill \textit{Conclusión}
    \item $((\varphi_1 \vee \varphi_2) \vee \varphi_3)$ \hfill \textit{Tesis 1, División por casos (1, 5, 15)}
    \item $(\varphi_1 \vee (\varphi_2 \vee \varphi_3)) \rightarrow ((\varphi_1 \vee \varphi_2) \vee \varphi_3)$ \hfill \textit{Conclusión}
\end{enumerate}

\end{proof}
\end{theorem}

\section*{Combo 4}

\begin{lemma}[Propiedades básicas de la deducción]
Sea $(\Sigma, \tau)$ una teoría.
\begin{enumerate}[label=(\arabic*)]
    \item (\textit{Uso de teoremas}) Si $(\Sigma, \tau) \vdash \varphi_1, \ldots, \varphi_n$ y $(\Sigma \cup \{\varphi_1, \ldots, \varphi_n\}, \tau) \vdash \varphi$, entonces $(\Sigma, \tau) \vdash \varphi$.
    \item Supongamos $(\Sigma, \tau) \vdash \varphi_1, \ldots, \varphi_n$. Si $R$ es una regla distinta de \textsc{Generalización} y \textsc{Elección} y $\varphi$ se deduce de $\varphi_1, \ldots, \varphi_n$ por la regla $R$, entonces $(\Sigma, \tau) \vdash \varphi$.
    \item $(\Sigma, \tau) \vdash (\varphi \to \psi)$ si y sólo si $(\Sigma \cup \{\varphi\}, \tau) \vdash \psi$.
\end{enumerate}
\end{lemma}

\begin{theorem}
Sea $(L, s, i, c, 0, 1)$ un álgebra de Boole y sean $a, b \in B$. Se tiene que:
\begin{enumerate}[label=(\arabic*)]
    \item $(a i b)^c = a^c s b^c$
    \item $a i b = 0$ si y sólo si $b \le a^c$
\end{enumerate}
\begin{proof}

\noindent\textbf{(1)} \[
(a \, i \, b)^c = a^c \, s \, b^c.
\]

Sea $(L,s,i,c,0,1)$ un álgebra de Boole, es decir, un reticulado complementado distributivo.  
En un álgebra de Boole vale la distributividad:
\[
x \, s \, (y \, i \, z) = (x \, s \, y) \, i \, (x \, s \, z),
\quad
x \, i \, (y \, s \, z) = (x \, i \, y) \, s \, (x \, i \, z).
\]

Queremos probar que $(a \, i \, b)^c$ cumple las propiedades de complemento de $a^c \, s \, b^c$.  
Para ello verificamos:

\[
(a \, i \, b) \, s \, (a^c \, s \, b^c) = 1
\quad\text{y}\quad
(a \, i \, b) \, i \, (a^c \, s \, b^c) = 0.
\]

\underline{Primero, la unión da $1$:}
\[
\begin{aligned}
(a \, i \, b) \, s \, (a^c \, s \, b^c)
&= \big((a \, s \, a^c) \, s \, (b \, s \, b^c)\big) &&\text{(por distributividad)}\\
&= (1 \, s \, 1) = 1.
\end{aligned}
\]

\underline{Luego, la intersección da $0$:}
\[
\begin{aligned}
(a \, i \, b) \, i \, (a^c \, s \, b^c)
&= \big((a \, i \, a^c) \, s \, (b \, i \, b^c)\big) &&\text{(por distributividad)}\\
&= (0 \, s \, 0) = 0.
\end{aligned}
\]

Por unicidad del complemento, se tiene entonces:
\[
(a \, i \, b)^c = a^c \, s \, b^c.
\]

\medskip

\noindent\textbf{(2)} Queremos probar que
\[
a \, i \, b = 0 \quad\Longleftrightarrow\quad b \leq a^c.
\]

\underline{($\Rightarrow$)}  
Supongamos $a i b = 0$. Se tiene
\[
\begin{aligned}
b &= (b i a) \, s \, (b i a^c) \quad \text{(por lema anterior)} \\
  &= (a i b) \, s \, (b_i a^c) \\
  &= 0 \, s \, (b_i a^c) \\
  &= (b i a^c) \leq a^c,
\end{aligned}
\]
por lo cual $b \leq a^c$.

\medskip

\underline{($\Leftarrow$)}  
\noindent
Supongamos ahora $b \leq a^c$.  
Ya que $a \leq a$, por lema de "ser menor que infimo", aplicado al reticulado par $(B, \leq)$, nos dice que
\[
a i b \leq a i a^c.
\]
Ya que $a i a^c = 0$, obtenemos que
\[
a i b = 0.
\]
\end{proof}
\end{theorem}

\begin{lemma}
Sean $(L, s, i)$ y $(L', s', i')$ reticulados terna y sean $(L, \le)$ y $(L', \le')$ los posets asociados. Sea $F: L \to L'$ una función. Entonces $F$ es un isomorfismo de $(L, s, i)$ en $(L', s', i')$ si y sólo si $F$ es un isomorfismo de $(L, \le)$ en $(L', \le')$.
\medskip
\begin{proof}
Sea $F: L \to L'$ un isomorfismo de posets, es decir:
\begin{enumerate}[label=\alph*)]
    \item $F$ es biyectiva,
    \item para todo $x, y \in L$, se cumple que 
    \[
    x \leq y \iff F(x) \leq' F(y).
    \]
\end{enumerate}

\noindent
Queremos probar que $F$ es un isomorfismo de reticulados terna, 
es decir, que además preserva las operaciones:
\[
F(x \, s \, y) = F(x) \, s' \, F(y),
\qquad
F(x \, i \, y) = F(x) \, i' \, F(y),
\]
para todo $x, y \in L$.

Por el teorema de Dedekind sabemos que 
\[
a \, s \, b = \sup\{a, b\} \text{ en } (L, \leq).
\]
Entonces $a \leq a \, s \, b$ y $b \leq a \, s \, b$.

\medskip
\noindent
Sean $x, y \in L$ y definamos $z := x \, s \, y$ en $L$. 
Queremos probar que 
\[
F(z) = F(x) \, s' \, F(y).
\]

\subsubsection*{1. $F(z)$ es cota superior de $\{F(x), F(y)\}$}

Puesto que $z$ es cota superior de $\{x, y\}$ en $L$, se tiene $x \leq z$ y $y \leq z$.  
Por la hipótesis de isomorfismo, $F(x) \leq' F(z)$ y $F(y) \leq' F(z)$.  
Por tanto, $F(z)$ es cota superior de $\{F(x), F(y)\}$ en $L'$.

\subsubsection*{2. $F(z)$ es la menor cota superior}

Sea $w' \in L'$ cualquier cota superior de $\{F(x), F(y)\}$; es decir, 
$F(x) \leq' w'$ y $F(y) \leq' w'$.  
Como $F$ es biyectiva, existe $w \in L$ tal que $F(w) = w'$.

\noindent
Usando la reflexión del orden (isomorfismo), obtenemos $x \leq w$ y $y \leq w$.  
Entonces $z = \sup\{x, y\} \leq w$.  
Aplicando $F$ y usando que $F$ preserva el orden, se tiene 
\[
F(z) \leq' F(w) = w'.
\]

\noindent
Esto muestra que cualquier cota superior $w'$ de $\{F(x), F(y)\}$ domina a $F(z)$.  
Por la definición de supremo, $F(z)$ es la menor cota superior, es decir:
\[
F(z) = \sup_{L'} \{F(x), F(y)\}.
\]

\noindent
Por la definición de la operación $s'$ en el reticulado terna asociado a $(L', \leq')$, se cumple:
\[
\sup_{L'}\{F(x), F(y)\} = F(x) \, s' \, F(y).
\]
Por tanto,
\[
F(x \, s \, y) = F(z) = F(x) \, s' \, F(y),
\]
como queríamos.

\medskip
\noindent
La prueba para el ínfimo es dual.
\end{proof}
\end{lemma}

\section*{Combo 5}

\begin{theorem}[de Completitud]
Sea $T = (\Sigma, \tau)$ una teoría de primer orden. Si $T \vDash \varphi$, entonces $T \vdash \varphi$. \\
Haga sólo el caso en que $\tau$ tiene una cantidad infinita de nombres de constantes que no ocurren en las sentencias de $\Sigma$. En la exposición de la prueba no es necesario que demuestre los ítems (1) y (5).
\end{theorem}

\section*{Combo 6}

\begin{theorem}[de Completitud]
Sea $T = (\Sigma, \tau)$ una teoría de primer orden. Si $T \vDash \varphi$, entonces $T \vdash \varphi$. \\
Haga sólo el caso en que $\tau$ tiene una cantidad infinita de nombres de constantes que no ocurren en las sentencias de $\Sigma$. En la exposición de la prueba no es necesario que demuestre los ítems (1), (2), (3) y (4).
\end{theorem}

\section*{Combo 7}

\begin{lemma}[Propiedades básicas de la deducción]
Sea $(\Sigma, \tau)$ una teoría.
\begin{enumerate}[label=(\arabic*)]
    \item (\textit{Uso de teoremas}) Si $(\Sigma, \tau) \vdash \varphi_1, \ldots, \varphi_n$ y $(\Sigma \cup \{\varphi_1, \ldots, \varphi_n\}, \tau) \vdash \varphi$, entonces $(\Sigma, \tau) \vdash \varphi$.
    \item Supongamos $(\Sigma, \tau) \vdash \varphi_1, \ldots, \varphi_n$. Si $R$ es una regla distinta de \textsc{Generalización} y \textsc{Elección} y $\varphi$ se deduce de $\varphi_1, \ldots, \varphi_n$ por la regla $R$, entonces $(\Sigma, \tau) \vdash \varphi$.
    \item $(\Sigma, \tau) \vdash (\varphi \to \psi)$ si y sólo si $(\Sigma \cup \{\varphi\}, \tau) \vdash \psi$.
\end{enumerate}
\end{lemma} (Ver combo 4.1)

\begin{lemma}
Sea $(L, s, i)$ un reticulado terna y sea $\theta$ una congruencia de $(L, s, i)$. Entonces:
\begin{enumerate}[label=(\arabic*)]
    \item $(L / \theta, \tilde{s}, \tilde{i})$ es un reticulado terna.
    \item El orden parcial $\tilde{\le}$ asociado al reticulado terna $(L / \theta, \tilde{s}, \tilde{i})$ cumple:
    \[
    x / \theta  \tilde{\le}  y / \theta \quad \text{ssi} \quad y \theta (x s y).
    \]
\end{enumerate}
\begin{proof}
\noindent\textbf{(1) $(L/\theta,\widetilde{s},\widetilde{\imath})$ es un reticulado terna.}

Por hipótesis $\theta$ es una congruencia, por lo tanto las operaciones
\[
x/\theta\ \widetilde{s}\ y/\theta := (x s y)/\theta,\qquad
x/\theta\ \widetilde{\imath}\ y/\theta := (x \imath y)/\theta
\]
están bien definidas (la condición de congruencia garantiza que la clase cociente no depende de los representantes).

Queda verificar las identidades (I1)--(I7) para las operaciones $\widetilde{s},\widetilde{\imath}$ en $L/\theta$. Tomaremos representantes y usaremos que $(L,s,i)$ satisface (I1)--(I7).

\begin{itemize}
  \item (I1) Identidad idempotente:
  \[
    (x/\theta)\ \widetilde{s}\ (x/\theta) = (x s x)/\theta = x/\theta,
  \]
  porque $x s x=x$ en $L$. De manera análoga $(x/\theta)\ \widetilde{\imath}\ (x/\theta)=x/\theta$.

  \item (I2) Conmutatividad de $\widetilde{s}$:
  \[
    (x/\theta)\ \widetilde{s}\ (y/\theta) = (x s y)/\theta = (y s x)/\theta = (y/\theta)\ \widetilde{s}\ (x/\theta).
  \]
  Igual para $\widetilde{\imath}$ usando la conmutatividad de $i$ en $L$.

  \item (I4) Asociatividad de $\widetilde{s}$:
  \[
    \big((x/\theta)\ \widetilde{s}\ (y/\theta)\big)\ \widetilde{s}\ (z/\theta)
    = ((x s y) s z)/\theta
    = (x s (y s z))/\theta
    = (x/\theta)\ \widetilde{s}\ \big((y/\theta)\ \widetilde{s}\ (z/\theta)\big).
  \]
  Análogo para $\widetilde{\imath}$ por (I5).

  \item (I6) Absorción:
  \[
    (x/\theta)\ \widetilde{s}\ \big((x/\theta)\ \widetilde{\imath}\ (y/\theta)\big)
    = (x s (x \imath y))/\theta
    = x/\theta,
  \]
  porque en $L$ vale $x s (x \imath y)=x$. La otra ley de absorción (I7) se verifica igual.
\end{itemize}

Por lo tanto $(L/\theta,\widetilde{s},\widetilde{\imath})$ satisface (I1)--(I7), es decir es un reticulado terna.

\medskip

\noindent\textbf{(2) Relación de orden $\widetilde{\leq}$ en el cociente.}

\noindent
Por definición de $\widetilde{\leq}$ tenemos que 
\[
x/\theta \widetilde{\leq} y/\theta \ \text{sii}\ y/\theta = x/\theta \widetilde{s} y/\theta.
\]
Pero 
\[
x/\theta \widetilde{s} y/\theta = (x s y)/\theta 
\quad \text{(por definición de $\widetilde{s}$)},
\]
por lo cual tenemos que 
\[
x/\theta \widetilde{\leq} y/\theta \ \text{sii}\ y/\theta = (x\, s\, y)/\theta.
\]
\end{proof}
\end{lemma}

\begin{lemma}
Sean $(L, s, i)$ y $(L', s', i')$ reticulados terna y sean $(L, \le)$ y $(L', \le')$ los posets asociados. Sea $F: L \to L'$ una función. Entonces $F$ es un isomorfismo de $(L, s, i)$ en $(L', s', i')$ si y sólo si $F$ es un isomorfismo de $(L, \le)$ en $(L', \le')$.
\end{lemma} (Ver combo 4.3)

\section*{Combo 8}

\begin{lemma}
Supongamos que $F: A \to B$ es un isomorfismo. Sea $\varphi =_{d} \varphi(v_1, \ldots, v_n) \in F_\tau$. Entonces:
\[
A \vDash \varphi[a_1, a_2, \ldots, a_n] \text{ ssi } B \vDash \varphi[F(a_1), F(a_2), \ldots, F(a_n)]
\]
para cada $a_1, a_2, \ldots, a_n \in A$.
\begin{proof}
\noindent
Haremos la prueba por inducción en $k$.

\medskip
También, en esta prueba se usará sin demostrar el siguiente lema:

\begin{quote}
\textbf{Lema 5.}  
Si $F$ es un homomorfismo entonces
\[
F(t^A[a_1, \ldots, a_n]) = t^B[F(a_1), \ldots, F(a_n)]
\]
para cada $t \in T_\tau$ y $a_1, \ldots, a_n \in A$.
\end{quote}

\bigskip
\noindent
\textbf{Caso base:} $\varphi \in F^\tau_0$.  
Por el lema de menú para fórmulas, $\varphi$ puede tener las siguientes formas:
\[
\varphi = (s = t) \quad \text{con } s,t \in T_\tau, 
\qquad \text{o bien} \qquad
\varphi = r(t_1, \ldots, t_n),\; n \ge 1,\; r \in R_n,\; t_1, \ldots, t_n \in T_\tau.
\]

\noindent
Si $\varphi = (s = t)$ entonces:
\[
A \vDash \varphi[a_1, \ldots, a_n] \text{ sii } s^A[a_1, \ldots, a_n] = t^A[a_1, \ldots, a_n].
\]
Luego, como $F$ es un isomorfismo,
\[
A \vDash \varphi[a_1, \ldots, a_n] \text{ sii } F(s^A[a_1, \ldots, a_n]) = F(t^A[a_1, \ldots, a_n]).
\]
Por el Lema 5,
\[
A \vDash \varphi[a_1, \ldots, a_n] \text{ sii } s^B[F(a_1), \ldots, F(a_n)] = t^B[F(a_1), \ldots, F(a_n)].
\]
Por definición,
\[
s^B[F(a_1), \ldots, F(a_n)] = t^B[F(a_1), \ldots, F(a_n)] 
\text{ sii } B \vDash \varphi[F(a_1), \ldots, F(a_n)].
\]

\noindent
El caso $\varphi = r(t_1, \ldots, t_n)$ es similar:
\[
A \vDash \varphi[a_1, \ldots, a_n] \text{ sii } (t_1^A[a_1, \ldots, a_n], \ldots, t_n^A[a_1, \ldots, a_n]) \in r^A.
\]
Como $F$ es isomorfismo,
\[
A \vDash \varphi[a_1, \ldots, a_n] \text{ sii } (F(t_1^A[a_1, \ldots, a_n]), \ldots, F(t_n^A[a_1, \ldots, a_n])) \in r^B.
\]
Por el Lema 5,
\[
A \vDash \varphi[a_1, \ldots, a_n] \text{ sii } (t_1^B[F(a_1), \ldots, F(a_n)], \ldots, t_n^B[F(a_1), \ldots, F(a_n)]) \in r^B.
\]
Y por definición,
\[
(t_1^B[F(a_1), \ldots, F(a_n)], \ldots, t_n^B[F(a_1), \ldots, F(a_n)]) \in r^B
\text{ sii } B \vDash \varphi[F(a_1), \ldots, F(a_n)].
\]
Por lo que queda probado el caso base.

\bigskip
\noindent
\textbf{Caso inductivo:} $\varphi \in F^\tau_{k+1}$.

\medskip
\textbf{Hipótesis inductiva (HI):}  
Si $\varphi \in F^\tau_k$ entonces
\[
A \vDash \varphi[a_1, \ldots, a_n] \iff B \vDash \varphi[F(a_1), \ldots, F(a_n)]
\quad \forall a_1, \ldots, a_n \in A.
\]

Si $\varphi \in F_\tau^k$, entonces claramente se cumple la propiedad, por lo que suponemos $\varphi \in F_\tau^{k+1} - F_\tau^k$.  
Ahora, por el lema de menú para fórmulas, tenemos distintos casos.

\medskip
\noindent
\textbf{Caso} $\varphi = (\varphi_1 \lor \varphi_2)$, con $\varphi_1, \varphi_2 \in F_\tau^k$.
\[
\begin{aligned}
A \vDash \varphi[a_1, \ldots, a_n]
&\text{ sii } A \vDash \varphi_1[a_1, \ldots, a_n] \text{ o } A \vDash \varphi_2[a_1, \ldots, a_n] && \text{(def. de $\vDash$)} \\
&\text{ sii } B \vDash \varphi_1[F(a_1), \ldots, F(a_n)] \text{ o } B \vDash \varphi_2[F(a_1), \ldots, F(a_n)] && \text{(HI)} \\
&\text{ sii } B \vDash \varphi[F(a_1), \ldots, F(a_n)] && \text{(def. de $\vDash$)}.
\end{aligned}
\]

Los casos $\varphi = (\varphi_1 \land \varphi_2)$, $\varphi = (\varphi_1 \Rightarrow \varphi_2)$, 
$\varphi = (\varphi_1 \Leftrightarrow \varphi_2)$ y $\varphi = \neg \varphi_1$ son análogos.

\medskip
\noindent
\textbf{Caso} $\varphi = \forall x_j \varphi_1$, con $\varphi_1 \in F_\tau^k$.  
Por definición:
\[
A \vDash \varphi[a_1, \ldots, a_n] \text{ sii } A \vDash \varphi_1[a_1, \ldots, a, \ldots, a_n] 
\quad \text{para todo } a \in A.
\]
Por la HI:
\[
A \vDash \varphi[a_1, \ldots, a_n] \text{ sii } B \vDash \varphi_1[F(a_1), \ldots, F(a), \ldots, F(a_n)] 
\quad \text{para todo } a \in A.
\]
Luego, como $F$ es isomorfismo, $\operatorname{Im}(F) = B$, entonces:
\[
A \vDash \varphi[a_1, \ldots, a_n] \text{ sii } B \vDash \varphi_1[F(a_1), \ldots, b, \ldots, F(a_n)] 
\quad \text{para todo } b \in B.
\]
Finalmente, por definición:
\[
A \vDash \varphi[a_1, \ldots, a_n] \text{ sii } B \vDash \varphi[F(a_1), \ldots, F(a_n)].
\]

El caso $\varphi = \exists x_j \varphi_1$ es análogo.
\end{proof}
\end{lemma}

\begin{lemma}
Sean $(P, \le)$ y $(P', \le')$ posets. Supongamos que $F$ es un isomorfismo de $(P, \le)$ en $(P', \le')$.
\begin{enumerate}[label=(\alph*)]
    \item Para cada $S \subseteq P$ y cada $a \in P$, se tiene que $a$ es cota superior (resp. inferior) de $S$ si y sólo si $F(a)$ es cota superior (resp. inferior) de $F(S)$.
    \item Para cada $S \subseteq P$, se tiene que existe $\sup(S)$ si y sólo si existe $\sup(F(S))$ y en tal caso se cumple que:
    \[
    F(\sup(S)) = \sup(F(S)).
    \]
\end{enumerate}
\begin{proof}
(a) Supongamos que $a$ es cota superior de $S$. 
Veamos que entonces $F(a)$ es cota superior de $F(S)$. 
Sea $x \in F(S)$. Sea $s \in S$ tal que $x = F(s)$. 
Ya que $s \leq a$, tenemos que $x = F(s) \leq' F(a)$.

Supongamos ahora que $F(a)$ es cota superior de $F(S)$ y veamos que entonces $a$ es cota superior de $S$. 
Sea $s \in S$. Ya que $F(s) \leq' F(a)$, tenemos que 
\[
s = F^{-1}(F(s)) \leq F^{-1}(F(a)) = a.
\]

(b) Supongamos que existe $\sup(S)$. 
Veamos entonces que $F(\sup(S))$ es el supremo de $F(S)$. 
Por (e), $F(\sup(S))$ es cota superior de $F(S)$. 
Supongamos que $b$ es cota superior de $F(S)$. 
Entonces $F^{-1}(b)$ es cota superior de $S$, 
por lo cual $\sup(S) \leq F^{-1}(b)$, 
produciendo $F(\sup(S)) \leq' b$.

En forma análoga, se ve que si existe $\sup(F(S))$, 
entonces $F^{-1}(\sup(F(S)))$ es el supremo de $S$.
\end{proof}
\end{lemma}

\end{document}

