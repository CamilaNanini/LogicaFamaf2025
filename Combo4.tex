\documentclass[a4paper,12pt]{article}
\usepackage[spanish]{babel}
\usepackage[utf8]{inputenc}
\usepackage[T1]{fontenc}
\usepackage{amsmath, amssymb, amsthm}
\usepackage{enumitem}

\title{Combos de Definiciones}
\author{Camila Nanini}

\newtheorem{theorem}{Teorema}
\newtheorem{lemma}{Lema}
\newcommand{\deduce}[2]{(#1, #2) \vdash}
\newcommand{\incon}{es inconsistente}

\begin{document}

\maketitle

\section*{Combo 4}

\begin{lemma}[Propiedades básicas de la deducción]
Sea $(\Sigma, \tau)$ una teoría.
\begin{enumerate}[label=(\arabic*)]
    \item (\textit{Uso de teoremas}) Si $(\Sigma, \tau) \vdash \varphi_1, \ldots, \varphi_n$ y $(\Sigma \cup \{\varphi_1, \ldots, \varphi_n\}, \tau) \vdash \varphi$, entonces $(\Sigma, \tau) \vdash \varphi$.
    \item Supongamos $(\Sigma, \tau) \vdash \varphi_1, \ldots, \varphi_n$. Si $R$ es una regla distinta de \textsc{Generalización} y \textsc{Elección} y $\varphi$ se deduce de $\varphi_1, \ldots, \varphi_n$ por la regla $R$, entonces $(\Sigma, \tau) \vdash \varphi$.
    \item $(\Sigma, \tau) \vdash (\varphi \to \psi)$ si y sólo si $(\Sigma \cup \{\varphi\}, \tau) \vdash \psi$.
\end{enumerate}
\end{lemma}

\begin{theorem}
Sea $(L, s, i, c, 0, 1)$ un álgebra de Boole y sean $a, b \in B$. Se tiene que:
\begin{enumerate}[label=(\arabic*)]
    \item $(a i b)^c = a^c s b^c$
    \item $a i b = 0$ si y sólo si $b \le a^c$
\end{enumerate}
\begin{proof}

\noindent\textbf{(1)} \[
(a \, i \, b)^c = a^c \, s \, b^c.
\]

Sea $(L,s,i,c,0,1)$ un álgebra de Boole, es decir, un reticulado complementado distributivo.  
En un álgebra de Boole vale la distributividad:
\[
x \, s \, (y \, i \, z) = (x \, s \, y) \, i \, (x \, s \, z),
\quad
x \, i \, (y \, s \, z) = (x \, i \, y) \, s \, (x \, i \, z).
\]

Queremos probar que $(a \, i \, b)^c$ cumple las propiedades de complemento de $a^c \, s \, b^c$.  
Para ello verificamos:

\[
(a \, i \, b) \, s \, (a^c \, s \, b^c) = 1
\quad\text{y}\quad
(a \, i \, b) \, i \, (a^c \, s \, b^c) = 0.
\]

\underline{Primero, la unión da $1$:}
\[
\begin{aligned}
(a \, i \, b) \, s \, (a^c \, s \, b^c)
&= \big((a \, s \, a^c) \, s \, (b \, s \, b^c)\big) &&\text{(por distributividad)}\\
&= (1 \, s \, 1) = 1.
\end{aligned}
\]

\underline{Luego, la intersección da $0$:}
\[
\begin{aligned}
(a \, i \, b) \, i \, (a^c \, s \, b^c)
&= \big((a \, i \, a^c) \, s \, (b \, i \, b^c)\big) &&\text{(por distributividad)}\\
&= (0 \, s \, 0) = 0.
\end{aligned}
\]

Por unicidad del complemento, se tiene entonces:
\[
(a \, i \, b)^c = a^c \, s \, b^c.
\]

\medskip

\noindent\textbf{(2)} Queremos probar que
\[
a \, i \, b = 0 \quad\Longleftrightarrow\quad b \leq a^c.
\]

\underline{($\Rightarrow$)}  
Supongamos $a i b = 0$. Se tiene
\[
\begin{aligned}
b &= (b i a) \, s \, (b i a^c) \quad \text{(por lema anterior)} \\
  &= (a i b) \, s \, (b_i a^c) \\
  &= 0 \, s \, (b_i a^c) \\
  &= (b i a^c) \leq a^c,
\end{aligned}
\]
por lo cual $b \leq a^c$.

\medskip

\underline{($\Leftarrow$)}  
\noindent
Supongamos ahora $b \leq a^c$.  
Ya que $a \leq a$, por lema de "ser menor que infimo", aplicado al reticulado par $(B, \leq)$, nos dice que
\[
a i b \leq a i a^c.
\]
Ya que $a i a^c = 0$, obtenemos que
\[
a i b = 0.
\]
\end{proof}
\end{theorem}

\begin{lemma}
Sean $(L, s, i)$ y $(L', s', i')$ reticulados terna y sean $(L, \le)$ y $(L', \le')$ los posets asociados. Sea $F: L \to L'$ una función. Entonces $F$ es un isomorfismo de $(L, s, i)$ en $(L', s', i')$ si y sólo si $F$ es un isomorfismo de $(L, \le)$ en $(L', \le')$.
\medskip
\begin{proof}
Sea $F: L \to L'$ un isomorfismo de posets, es decir:
\begin{enumerate}[label=\alph*]
    \item $F$ es biyectiva,
    \item para todo $x, y \in L$, se cumple que 
    \[
    x \leq y \iff F(x) \leq' F(y).
    \]
\end{enumerate}

\noindent
Queremos probar que $F$ es un isomorfismo de reticulados terna, 
es decir, que además preserva las operaciones:
\[
F(x \, s \, y) = F(x) \, s' \, F(y),
\qquad
F(x \, i \, y) = F(x) \, i' \, F(y),
\]
para todo $x, y \in L$.

Por el teorema de Dedekind sabemos que 
\[
a \, s \, b = \sup\{a, b\} \text{ en } (L, \leq).
\]
Entonces $a \leq a \, s \, b$ y $b \leq a \, s \, b$.

\medskip
\noindent
Sean $x, y \in L$ y definamos $z := x \, s \, y$ en $L$. 
Queremos probar que 
\[
F(z) = F(x) \, s' \, F(y).
\]

\subsubsection*{1. $F(z)$ es cota superior de $\{F(x), F(y)\}$}

Puesto que $z$ es cota superior de $\{x, y\}$ en $L$, se tiene $x \leq z$ y $y \leq z$.  
Por la hipótesis de isomorfismo, $F(x) \leq' F(z)$ y $F(y) \leq' F(z)$.  
Por tanto, $F(z)$ es cota superior de $\{F(x), F(y)\}$ en $L'$.

\subsubsection*{2. $F(z)$ es la menor cota superior}

Sea $w' \in L'$ cualquier cota superior de $\{F(x), F(y)\}$; es decir, 
$F(x) \leq' w'$ y $F(y) \leq' w'$.  
Como $F$ es biyectiva, existe $w \in L$ tal que $F(w) = w'$.

\noindent
Usando la reflexión del orden (isomorfismo), obtenemos $x \leq w$ y $y \leq w$.  
Entonces $z = \sup\{x, y\} \leq w$.  
Aplicando $F$ y usando que $F$ preserva el orden, se tiene 
\[
F(z) \leq' F(w) = w'.
\]

\noindent
Esto muestra que cualquier cota superior $w'$ de $\{F(x), F(y)\}$ domina a $F(z)$.  
Por la definición de supremo, $F(z)$ es la menor cota superior, es decir:
\[
F(z) = \sup_{L'} \{F(x), F(y)\}.
\]

\noindent
Por la definición de la operación $s'$ en el reticulado terna asociado a $(L', \leq')$, se cumple:
\[
\sup_{L'}\{F(x), F(y)\} = F(x) \, s' \, F(y).
\]
Por tanto,
\[
F(x \, s \, y) = F(z) = F(x) \, s' \, F(y),
\]
como queríamos.

\medskip
\noindent
La prueba para el ínfimo es dual.
\end{proof}
\end{lemma}


\section*{Auxiliares de Demostraciones}

\begin{enumerate}
    \item Teorema de Dedekind: Sea $(L, s, i)$ un reticulado terna. La relación binaria definida por
        \[
        x \le y \ \text{si y sólo si} \ x \, s \, y = y
        \]
        es un orden parcial sobre $L$, para el cual se cumple que:
        \[
        \sup\{x, y\} = x \, s \, y \quad \text{y} \quad \inf\{x, y\} = x \, i \, y
        \]
        cualesquiera sean $x, y \in L$.
    \item Unicidad del complemento: Sea $(L, s, i, 0, 1)$ un reticulado acotado. 
    Si $(L, s, i, 0, 1)$ es distributivo, entonces todo elemento tiene a lo sumo un complemento. 
    Es decir, si 
    \[
    x \, s \, u = x \, s \, v = 1 
    \quad \text{y} \quad 
    x \, i \, u = x \, i \, v = 0,
    \]
    entonces $u = v$, cualesquiera sean $x, u, v \in L$.
    \item "Lema anterior": Sea $(B, s, i, c, 0, 1)$ un álgebra de Boole. 
    Cualesquiera sean $x, y \in B$, se tiene que 
    \[
    y = (y \, i \, x) \, s \, (y \, i \, x^c).
    \]

\end{enumerate}


\end{document}

