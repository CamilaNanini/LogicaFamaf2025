\documentclass[a4paper,12pt]{article}
\usepackage[spanish]{babel}
\usepackage[utf8]{inputenc}
\usepackage[T1]{fontenc}
\usepackage{amsmath, amssymb, amsthm}
\usepackage{enumitem}

\title{Combo 4}
\author{Camila Nanini}

\newtheorem{theorem}{Teorema}
\newtheorem{lemma}{Lema}
\newcommand{\deduce}[2]{(#1, #2) \vdash}
\newcommand{\incon}{es inconsistente}

\begin{document}

\maketitle

\section*{Combo 4}

\begin{lemma}[Propiedades básicas de la deducción]
Sea $(\Sigma, \tau)$ una teoría.
\begin{enumerate}[label=(\arabic*)]
    \item (\textit{Uso de teoremas}) Si $(\Sigma, \tau) \vdash \varphi_1, \ldots, \varphi_n$ y $(\Sigma \cup \{\varphi_1, \ldots, \varphi_n\}, \tau) \vdash \varphi$, entonces $(\Sigma, \tau) \vdash \varphi$.
    \item Supongamos $(\Sigma, \tau) \vdash \varphi_1, \ldots, \varphi_n$. Si $R$ es una regla distinta de \textsc{Generalización} y \textsc{Elección} y $\varphi$ se deduce de $\varphi_1, \ldots, \varphi_n$ por la regla $R$, entonces $(\Sigma, \tau) \vdash \varphi$.
    \item $(\Sigma, \tau) \vdash (\varphi \to \psi)$ si y sólo si $(\Sigma \cup \{\varphi\}, \tau) \vdash \psi$.
\end{enumerate}

\begin{proof}
(1) Nótese que basta con hacer el caso $n = 1$. El caso con $n \geq 2$ se obtiene aplicando $n$ veces el caso $n = 1$.  
Supongamos entonces que $(\Sigma, \tau) \vdash \varphi_1$ y $(\Sigma \cup \{\varphi_1\}, \tau) \vdash \varphi$.  
Sea $(\alpha_1, \ldots, \alpha_h, I_1, \ldots, I_h)$ una prueba formal de $\varphi_1$ en $(\Sigma, \tau)$.  
Sea $(\psi_1, \ldots, \psi_m, J_1, \ldots, J_m)$ una prueba formal de $\varphi$ en $(\Sigma \cup \{\varphi_1\}, \tau)$.  

Nótese que, por los Lemas Cambio de indice de hipotesis y Cambio de nombres de constante auxiliares, podemos suponer que estas dos pruebas no comparten ningún nombre de constante auxiliar y que tampoco comparten números asociados a hipótesis o tesis.  

Para cada $i = 1, \ldots, m$, definamos $\tilde{J}_i$ de la siguiente manera:

\begin{itemize}
    \item Si $J_i = \alpha\,\textsc{AxiomaPropio}$, con $\alpha \in \{\varepsilon\} \cup \{\textsc{Tesis}\bar{k} : k \in \mathbb{N}\}$ y $\psi_i = \varphi_1$, entonces $\tilde{J}_i = \alpha\,\textsc{Evocación}(\bar{h})$.
    \item Si $J_i = \alpha\,\textsc{AxiomaPropio}$, con $\alpha \in \{\varepsilon\} \cup \{\textsc{Tesis}\bar{k} : k \in \mathbb{N}\}$ y $\psi_i \notin \{\varphi_1\}$, entonces $\tilde{J}_i = \alpha\,\textsc{AxiomaPropio}$.
    \item Si $J_i = \alpha\,\textsc{AxiomaLógico}$, con $\alpha \in \{\varepsilon\} \cup \{\textsc{Tesis}\bar{k} : k \in \mathbb{N}\}$, entonces $\tilde{J}_i = \alpha\,\textsc{AxiomaLógico}$.
    \item Si $J_i = \alpha\,\textsc{Conclusión}$, con $\alpha \in \{\varepsilon\} \cup \{\textsc{Tesis}\bar{k} : k \in \mathbb{N}\}$, entonces $\tilde{J}_i = \alpha\,\textsc{Conclusión}$.
    \item Si $J_i = \textsc{Hipótesis}\bar{k}$, entonces $\tilde{J}_i = \textsc{Hipótesis}\bar{k}$.
    \item Si $J_i = \alpha R(\bar{l}_1, \ldots, \bar{l}_k)$, con $\alpha \in \{\varepsilon\} \cup \{\textsc{Tesis}\bar{k} : k \in \mathbb{N}\}$, entonces 
    \[
    \tilde{J}_i = \alpha R(\overline{l_1+h} , \ldots, \overline{l_k+h} ).
    \]
\end{itemize}

Es fácil chequear que 
\[
(\alpha_1, \ldots, \alpha_h, \psi_1, \ldots, \psi_m, I_1, \ldots, I_h, \tilde{J}_1, \ldots, \tilde{J}_m)
\]
es una prueba formal de $\varphi$ en $(\Sigma, \tau)$.

\vspace{1em}
(2) Nótese que:
\begin{align*}
1. &\quad \varphi_1 \quad \textsc{AxiomaPropio} \\
2. &\quad \varphi_2 \quad \textsc{AxiomaPropio} \\
\vdots \\
n. &\quad \varphi_n \quad \textsc{AxiomaPropio} \\
n+1. &\quad \varphi \quad R(\bar{1}, \ldots, \bar{n})
\end{align*}
es una prueba formal de $\varphi$ en $(\Sigma \cup \{\varphi_1, \ldots, \varphi_n\}, \tau)$, lo cual por (1) nos dice que $(\Sigma, \tau) \vdash \varphi$.

\vspace{1em}
(3) Supongamos $(\Sigma, \tau) \vdash (\varphi \to \psi)$. Entonces tenemos que $(\Sigma \cup \{\varphi\}, \tau) \vdash (\varphi \to \psi), \varphi$, lo cual por (2) nos dice que $(\Sigma \cup \{\varphi\}, \tau) \vdash \psi$.  

Supongamos ahora que $(\Sigma \cup \{\varphi\}, \tau) \vdash \psi$. Sea $(\varphi_1, \ldots, \varphi_n, J_1, \ldots, J_n)$ una prueba formal de $\psi$ en $(\Sigma \cup \{\varphi\}, \tau)$.  

Para cada $i = 1, \ldots, n$, definamos $\tilde{J}_i$ como sigue:

\begin{itemize}
    \item Si $\varphi_i = \varphi$ y $J_i = \alpha\,\textsc{AxiomaPropio}$, con $\alpha \in \{\varepsilon\} \cup \{\textsc{Tesis}\bar{k} : k \in \mathbb{N}\}$, entonces $\tilde{J}_i = \alpha\,\textsc{Evocación}(1)$.
    \item Si $\varphi_i \neq \varphi$ y $J_i = \alpha\,\textsc{AxiomaPropio}$, con $\alpha \in \{\varepsilon\} \cup \{\textsc{Tesis}\bar{k} : k \in \mathbb{N}\}$, entonces $\tilde{J}_i = \alpha\,\textsc{AxiomaPropio}$.
    \item Si $J_i = \alpha\,\textsc{AxiomaLógico}$, con $\alpha \in \{\varepsilon\} \cup \{\textsc{Tesis}\bar{k} : k \in \mathbb{N}\}$, entonces $\tilde{J}_i = \alpha\,\textsc{AxiomaLógico}$.
    \item Si $J_i = \alpha\,\textsc{Conclusión}$, con $\alpha \in \{\varepsilon\} \cup \{\textsc{Tesis}\bar{k} : k \in \mathbb{N}\}$, entonces $\tilde{J}_i = \alpha\,\textsc{Conclusión}$.
    \item Si $J_i = \textsc{Hipótesis}\bar{k}$, entonces $\tilde{J}_i = \textsc{Hipótesis}\bar{k}$.
    \item Si $J_i = \alpha R(\bar{l}_1, \ldots, \bar{l}_k)$, con $\alpha \in \{\varepsilon\} \cup \{\textsc{Tesis}\bar{k} : k \in \mathbb{N}\}$, entonces 
    \[
    \tilde{J}_i = \alpha R(\overline{l_1 + 1}, \ldots, \overline{l_k + 1}).
    \]
\end{itemize}

Sea $m$ tal que ninguna $J_i$ es igual a $\textsc{Hipótesis}\bar{m}$.  
Nótese que $\tilde{J}_n$ no es de la forma $\textsc{Tesis}\bar{k}\beta$ ni de la forma $\textsc{Hipótesis}\bar{k}$, por lo cual $\textsc{Tesis}\bar{m}\tilde{J}_n$ es una justificación.  

Es fácil chequear que:
\[
(\varphi, \varphi_1, \ldots, \varphi_n, (\varphi \to \psi), 
\textsc{Hipótesis}\bar{m}, \tilde{J}_1, \ldots, \tilde{J}_{n-1}, \textsc{Tesis}\bar{m}\tilde{J}_n, \textsc{Conclusión})
\]
es una prueba formal de $(\varphi \to \psi)$ en $(\Sigma, \tau)$.
\end{proof}

\end{lemma}

\begin{theorem}
Sea $(L, s, i, c, 0, 1)$ un álgebra de Boole y sean $a, b \in B$. Se tiene que:
\begin{enumerate}[label=(\arabic*)]
    \item $(a i b)^c = a^c s b^c$
    \item $a i b = 0$ si y sólo si $b \le a^c$
\end{enumerate}
\begin{proof}

\noindent\textbf{(1)} \[
(a \, i \, b)^c = a^c \, s \, b^c.
\]

Sea $(L,s,i,c,0,1)$ un álgebra de Boole, es decir, un reticulado complementado distributivo.  
En un álgebra de Boole vale la distributividad:
\[
x \, s \, (y \, i \, z) = (x \, s \, y) \, i \, (x \, s \, z),
\quad
x \, i \, (y \, s \, z) = (x \, i \, y) \, s \, (x \, i \, z).
\]

Queremos probar que $(a \, i \, b)^c$ cumple las propiedades de complemento de $a^c \, s \, b^c$.  
Para ello verificamos:

\[
(a \, i \, b) \, s \, (a^c \, s \, b^c) = 1
\quad\text{y}\quad
(a \, i \, b) \, i \, (a^c \, s \, b^c) = 0.
\]

\underline{Primero, la unión da $1$:}
\[
\begin{aligned}
(a \, i \, b) \, s \, (a^c \, s \, b^c)
&= \big((a \, s \, a^c) \, s \, (b \, s \, b^c)\big) &&\text{(por distributividad)}\\
&= (1 \, s \, 1) = 1.
\end{aligned}
\]

\underline{Luego, la intersección da $0$:}
\[
\begin{aligned}
(a \, i \, b) \, i \, (a^c \, s \, b^c)
&= \big((a \, i \, a^c) \, s \, (b \, i \, b^c)\big) &&\text{(por distributividad)}\\
&= (0 \, s \, 0) = 0.
\end{aligned}
\]

Por unicidad del complemento, se tiene entonces:
\[
(a \, i \, b)^c = a^c \, s \, b^c.
\]

\medskip

\noindent\textbf{(2)} Queremos probar que $a \, i \, b = 0 \quad\Longleftrightarrow\quad b \leq a^c.$

\underline{($\Rightarrow$)}  
Supongamos $a i b = 0$. Se tiene
\[
\begin{aligned}
b &= (b i a) \, s \, (b i a^c) \quad \text{(por lema anterior)} \\
  &= (a i b) \, s \, (b_i a^c) \\
  &= 0 \, s \, (b_i a^c) \\
  &= (b i a^c) \leq a^c,
\end{aligned}
\]
por lo cual $b \leq a^c$.

\medskip

\underline{($\Leftarrow$)}  
\noindent
Supongamos ahora $b \leq a^c$.  Entonces como $aib \leq b$ por transitividad
$aib \leq a^c$. Ahora $a \leq a$, por lo que $a i b \leq a i a^c$.
Ya que $a i a^c = 0$, obtenemos que $aib \leq 0$ y por axioma $0 \leq aib$, 
entonces $a i b = 0.$

\textbf{Pequeña demo del paso $a i b \leq a i a^c$} Supongamos \(a \, i \, b \le a^{c}\). Por definición de \(\le\),
\[
(a \, i \, b) \, i \, a^{c} = a \, i \, b. \tag{1}
\]

Queremos probar \(a \, i \, b \le a \, i \, a^{c}\), es decir demostrar
\[
(a \, i \, b) \, i \, (a \, i \, a^{c}) = a \, i \, b.
\]

\[
\begin{aligned}
(a \, i \, b)\, i \,(a \, i \, a^{c})
&= a\, i \,\bigl( b \, i \, (a \, i \, a^{c})\bigr)
    &&\text{(asociatividad / conmutatividad)}\\[4pt]
&= a\, i \,\bigl( (b \, i \, a)\, i \, a^{c}\bigr)
    &&\text{(asociatividad)}\\[4pt]
&= a\, i \,\bigl( (a \, i \, b)\, i \, a^{c}\bigr)
    &&\text{(conmutatividad \(b\, i \, a = a \, i \, b\))}\\[4pt]
&= a\, i \,(a \, i \, b)
    &&\text{(por (1), sustituyendo \((a \, i \, b)\, i \, a^{c} = a \, i \, b\))}\\[4pt]
&= (a \, i \, a)\, i \, b
    &&\text{(asociatividad)}\\[4pt]
&= a \, i \, b
    &&\text{(idempotencia \(a \, i \, a = a\)).}
\end{aligned}
\]

Hemos obtenido
\[
(a \, i \, b)\, i \,(a \, i \, a^{c}) = a \, i \, b,
\]
que por la definición de \(\le\) es precisamente \(a \, i \, b \le a \, i \, a^{c}\).


\end{proof}
\end{theorem}

\begin{lemma}
Sean $(L, s, i)$ y $(L', s', i')$ reticulados terna y sean $(L, \le)$ y $(L', \le')$ los posets asociados. Sea $F: L \to L'$ una función. Entonces $F$ es un isomorfismo de $(L, s, i)$ en $(L', s', i')$ si y sólo si $F$ es un isomorfismo de $(L, \le)$ en $(L', \le')$.
\medskip
\begin{proof}
\medskip
\noindent
$(\Leftarrow)$; 
Sea $F: L \to L'$ un isomorfismo de posets, es decir:
\begin{enumerate}[label=\alph*]
    \item $F$ es biyectiva,
    \item para todo $x, y \in L$, se cumple que $x \leq y \iff F(x) \leq' F(y)$
\end{enumerate}

\noindent
Queremos probar que $F$ es un isomorfismo de reticulados terna, 
es decir, que preserva las operaciones:
\[
F(x \, s \, y) = F(x) \, s' \, F(y),
\qquad
F(x \, i \, y) = F(x) \, i' \, F(y),
\]
para todo $x, y \in L$.

Por el teorema de Dedekind sabemos que $a \, s \, b = \sup\{a, b\} \text{ en } (L, \leq).$
Entonces $a \leq a \, s \, b$ y $b \leq a \, s \, b$.

\medskip
\noindent
Sean $x, y \in L$ y definamos $z := x \, s \, y$ en $L$. 
Queremos probar que 
\[
F(z) = F(x) \, s' \, F(y).
\]

\subsubsection*{1. $F(z)$ es cota superior de $\{F(x), F(y)\}$}

Puesto que $z$ es cota superior de $\{x, y\}$ en $L$, se tiene $x \leq z$ y $y \leq z$.  
Por la hipótesis de isomorfismo, $F(x) \leq' F(z)$ y $F(y) \leq' F(z)$.  
Por tanto, $F(z)$ es cota superior de $\{F(x), F(y)\}$ en $L'$.

\subsubsection*{2. $F(z)$ es la menor cota superior}

Sea $w' \in L'$ cualquier cota superior de $\{F(x), F(y)\}$; es decir, 
$F(x) \leq' w'$ y $F(y) \leq' w'$.  
Como $F$ es biyectiva, existe $w \in L$ tal que $F(w) = w'$.

\noindent
Usando la reflexión del orden (isomorfismo), obtenemos $x \leq w$ y $y \leq w$.  
Entonces $z = \sup\{x, y\} \leq w$.  
Aplicando $F$ y usando que $F$ preserva el orden, se tiene 
\[
F(z) \leq' F(w) = w'.
\]

\noindent
Esto muestra que cualquier cota superior $w'$ de $\{F(x), F(y)\}$ domina a $F(z)$.  

\medskip
\noindent
$(\Rightarrow)$; 
Supongamos que  F es un isomorfismo de (L, s, i) en (L', s', i'). 
Sean x, y $\in$ L tales que x $\le$ y. 
Tenemos que y = xsy, por lo cual F(y) = F(xsy) = F(x)s'F(y),
ocurriendo que $F(x) \le' F(y)$.
De forma similar se puede ver que $F^{-1}$ es también un homomorfismo de $(L', \le')$ en $(L, \le)$.

La prueba para el ínfimo es dual.
\end{proof}
\end{lemma}


\section*{Auxiliares de Demostraciones}

\begin{enumerate}
    \item Teorema de Dedekind: Sea $(L, s, i)$ un reticulado terna. La relación binaria definida por
        \[
        x \le y \ \text{si y sólo si} \ x \, s \, y = y
        \]
        es un orden parcial sobre $L$, para el cual se cumple que:
        \[
        \sup\{x, y\} = x \, s \, y \quad \text{y} \quad \inf\{x, y\} = x \, i \, y
        \]
        cualesquiera sean $x, y \in L$.
    \item Unicidad del complemento: Sea $(L, s, i, 0, 1)$ un reticulado acotado. 
    Si $(L, s, i, 0, 1)$ es distributivo, entonces todo elemento tiene a lo sumo un complemento. 
    Es decir, si 
    \[
    x \, s \, u = x \, s \, v = 1 
    \quad \text{y} \quad 
    x \, i \, u = x \, i \, v = 0,
    \]
    entonces $u = v$, cualesquiera sean $x, u, v \in L$.
    \item "Lema anterior": Sea $(B, s, i, c, 0, 1)$ un álgebra de Boole. 
    Cualesquiera sean $x, y \in B$, se tiene que 
    \[
    y = (y \, i \, x) \, s \, (y \, i \, x^c).
    \]
    \item Cambio de índice de hipótesis: Sea $(\varphi, J)$ una prueba formal de $\varphi$ en $(\Sigma, \tau)$.  
    Sea $m \in \mathbb{N}$ tal que $J_i \neq \text{HIPÓTESIS}\overline{m}$, para cada $i = 1, \ldots, n(\varphi)$.  
    Supongamos que $J_i = \text{HIPÓTESIS}\overline{k}$ y que $J_j = \text{TESIS}\overline{k} \, \alpha$, con $[\alpha]_1 \notin \text{Num}$.  
    Sea $\tilde{J}$ el resultado de reemplazar en $J$ la justificación $J_i$ por $\text{HIPÓTESIS}\overline{m}$  
    y reemplazar la justificación $J_j$ por $\text{TESIS}\overline{m} \, \alpha$.  
    Entonces $(\varphi, \tilde{J})$ es una prueba formal de $\varphi$ en $(\Sigma, \tau)$.
    \item Cambio de nombres de constantes auxiliares: Sea $(\varphi, J)$ una prueba formal de $\varphi$ en $(\Sigma, \tau)$.  
    Sea $C_1$ el conjunto de nombres de constantes auxiliares de $(\varphi, J)$.  
    Sea $e \in C_1$.  
    Sea $\tilde{e} \notin C \cup C_1$ tal que 
    \[
    (C \cup (C_1 - \{e\}) \cup \{\tilde{e}\}, F, R, a)
    \]
    es un tipo.  

    Sea $\tilde{\varphi_i}$ el resultado de reemplazar en $\varphi_i$ cada ocurrencia de $e$ por $\tilde{e}$.  

    Entonces 
    \[
    (\tilde{\varphi}_1, \ldots, \tilde{\varphi}_{n(\varphi)}, J)
    \]
    es una prueba formal de $\varphi$ en $(\Sigma, \tau)$.
\end{enumerate}

Axiomas elementales de reticulados terna:
\begin{enumerate}
    \item $\forall x \,(x \, s \, x = x)$
    \item $\forall x \,(x \, i \, x = x)$
    \item $\forall x \forall y \,(x \, s \, y = y \, s \, x)$
    \item $\forall x \forall y \,(x \, i \, y = y \, i \, x)$
    \item $\forall x \forall y \forall z \,((x \, s \, y) \, s \, z = x \, s \,(y \, s \, z))$
    \item $\forall x \forall y \forall z \,((x \, i \, y) \, i \, z = x \, i \,(y \, i \, z))$
    \item $\forall x \forall y \,(x \, s \,(x \, i \, y) = x)$
    \item $\forall x \forall y \,(x \, i \,(x \, s \, y) = x)$
\end{enumerate}

\end{document}

